对计算机而言,要编写程序就必须使用计算机语言。计算机语言是指编写程序时,根据事先定义的规则而写出的预定义语句集合。计算机语言经过多年的发展已经从机器语言演化到高级语言。

在计算机发展的早期,唯一的程序设计语言是机器语言。每台计算机有其自己的机器语言,这种机器语言由“0”和“1”序列组成。在第5章中,我们看到在一台原始假想的计算机中,我们需要用11行代码去读两个整数、把它们相加并输出结果。当用机器语言来写时,这些代码就成了11行二进制代码,每行16位,如表所示。

机器语言是计算机硬件唯一能理解的语言,它由具有两种状态的电子开关构成:关(表示0)或开(表示1)。

计算机唯一识别的语言是机器语言。

虽然用机器语言编写真实地表示了数据是如何被计算机操纵的。但它至少有两个缺点:首先,它依赖于计算机。如果使用不同的硬件,那么一台计算机的机器语言与另一台计算机的机器语言就不同。其次,用这种语言编写程序时非常单调乏味的,而且很难发现错误。现在我们将机器语言时代称为编程语言的第一代。

编程语言中接下来的演化是伴随着带符号或助记符的指令和地址代替二进制码而发生的。因为他们使用符号,所以这些语言首先被称为符号语言。这些助记符语言后来就被称为汇编语言。假想计算机用于替代机器语言的汇编语言显示在程序中。

尽管汇编语言大大提高了变成效率,但任然需要程序员在所使用的硬件上花费大部分精力。用符号语言编程也很枯燥,因为每条机器指令都必须单独编码。为了提高程序员效率以及从关注计算机转到关注解决的问题,促进了高级语言的发展。

高级语言可移植到许多不同的计算机,使得程序员能够将精力集中在应用程序上,而不是计算机结构的复杂性上。高级语言旨在使程序员摆脱汇编语言繁琐的细节。高级语言同汇编语言有一个共性:它们必须被转化为机器语言,这个转化过程被称为解释或编译。

数年来,人们开发了各种各样的语言,最著名的有BASIC、COBOL、Pascal、Ada、C、C++和Java。

当今程序通常是同一种高级语言来编写。为了在计算机上运行程序,程序需要被翻译成它要运行在其他的计算机的机器语言。高级语言程序被称为源程序。被翻译成的机器语言程序被称为目标程序。有两种方法用于翻译:编译和解释。

编译程序通常把整个源程序翻译成目标程序。

有些计算机语言使用解释器把源程序翻译成目标程序。解释是把源程序中的每一行翻译成目标程序中相应的行,并执行它的过程。但是,我们需要意识到在解释中的两种趋势:在java语言之前被有些程序使用的和java使用的解释。

在java语言之前的有些解释式语言使用一种被称为解释的第一种方法的解释过程,因为缺少其他任何名字,说以称为解释的第一种方法。在这种解释中,程序的每一行被翻译成其使用的计算机上的机器语言,该行机器语言被立即执行。如果在翻译或执行中有任何错误,过程就显示消息,其余的过程就被终止。程序需要被改正,再次从头解释和执行。第一种方法被看成是一种慢的过程,这就是大多数语言使用编译而不是解释的原因。

随着java的到来,一种新的解释过程就被引入了。java语言能向任何计算机移植。为了取得可移植性,源程序到目标程序的翻译分成两步进行:编译和解释。java源程序首先被编译,创建java的字节代码,字节代码看起来像机器语言中的代码,但不是任何特定计算机的目标代码,它是一种虚拟机的目标代码,该虚拟机称为java虚拟机或JVM。字节代码然后能被任何运次JVM模拟器的计算机编译或解释,也就是运行字节代码的计算机只需要JVM模拟器,而不是java编译器。

编译和解释的不同在于,编译在执行前翻译整个源代码,而解释一次只翻译和执行源代码中的一行。但是,两种方法都遵循图中显示的相同的翻译过程。

词法分析器一个符号一个符号地读取源代码,创建源语言中的助记符表。

词法分析器分析一组助记符,找出指令。

语义分析器检查语法分析器创建的句子,确保它们不含有二义性。计算机语言通常是无二义性的,这意味着这一步骤或者是在翻译器中被省略,或者其责任被最小化。

在无二义性指令被语义分析器创建之后,每条指令将转化为一组程序将要在其上运行的计算机的机器语言。这个是由代码生成器完成的。

当今计算机语言按照它们使用的解决问题的方法分类。因此,模式是计算机语言看待要解决问题的一种方式。计算机语言课分成4中模式:过程性、面向对象、函数式和声说式。

在过程模式中我们把程序看成是操纵被动对象的主动主体。我们日常生活中遇到许多被动对象:石头、书、灯等。一个被动对象本身不能发出一个动作,但它能从主动主体接收动作。

过程式模式下的程序就是主动主体,该主体使用称为数据或数据项的被动对象。作为被动对象的数据项存储在计算机的内存中,程序操作他们。为了操纵数据,主动主体发出动作,称之为过程。

我们需要把过程和 程序触发区分开。程序不定义过程,它只触发或调用过程。过程必须已经存在。

当使用过程式高级语言时,程序仅由许多过程调用构成,除此之外没有任何东西。这不是显而易见的,但即使使用像加法运算符这样的简单数学运算符时,我们也是正在使用一个过程调用一个已经编写的过程。

如果我们考虑过程和被作用于的对象,那么过程式模式的概念就变得更为简单,而且容易理解。这种模式的程序由三部分构成:对象创建部分、一组过程调用和每个过程的一组代码。有些过程在语言本身中已经被定义。通过组合这些代码,开发者可以建立新的过程。

面向对象模式处理活动对象,而不是被动对象。我们在日常生活中遇到许多活动对象:汽车、自动门、洗盘机。在这些对象上执行的动作都包含在这些对象那中:对象只需要接收合适的外部刺激来执行其中一个动作。

总体上,方法的格式与有些过程式语言中用的函数非常相似。每个方法有它的头、局部变量和语句。这就意味着我们对过程式语言所讨论的大多数特性都可以应用在为面向对象程序所写的方法上。换言之,我们可以认为面向对象语言实际上是带有新的理念和新的特性的过程式语言的扩展。

在面向对象模式中,作为本质,一个对象能从另一个对象继承。这个概念被称为继承性。当一般类被定义后,我们可以定义继承了一般类中一些特性的更具体的类,同时这些类具有一些新特性。

多态性因为他们使用符号,所以这些语言首先被称为符号语言。这些助记符语言后来就被称为汇编语言。假想计算机用于替代机器语言的汇编语言显示在程序中。

尽管汇编语言大大提高了变成效率,但任然需要程序员在所使用的硬件上花费大部分精力。用符号语言编程也很枯燥,因为每条机器指令都必须单独编码。为了提高程序员效率以及从关注计算机转到关注解决的问题,促进了高级语言的发展。

高级语言可移植到许多不同的计算机,使得程序员能够将精力集中在应用程序上,而不是计算机结构的复杂性上。高级语言旨在使程序员摆脱汇编语言繁琐的细节。高级语言同汇编语言有一个共性:它们必须被转化为机器语言,这个转化过程被称为解释或编译。

数年来,人们开发了各种各样的语言,最著名的有BASIC、COBOL、Pascal、Ada、C、C++和Java。

当今程序通常是同一种高级语言来编写。为了在计算机上运行程序,程序需要被翻译成它要运行在其他的计算机的机器语言。高级语言程序被称为源程序。被翻译成的机器语言程序被称为目标程序。有两种方法用于翻译:编译和解释。

编译程序通常把整个源程序翻译成目标程序。

有些计算机语言使用解释器把源程序翻译成目标程序。解释是把源程序中的每一行翻译成目标程序中相应的行,并执行它的过程。但是,我们需要意识到在解释中的两种趋势:在java语言之前被有些程序使用的和java使用的解释。

在java语言之前的有些解释式语言使用一种被称为解释的第一种方法的解释过程,因为缺少其他任何名字,说以称为解释的第一种方法。在这种解释中,程序的每一行被翻译成其使用的计算机上的机器语言,该行机器语言被立即执行。如果在翻译或执行中有任何错误,过程就显示消息,其余的过程就被终止。程序需要被改正,再次从头解释和执行。第一种方法被看成是一种慢的过程,这就是大多数语言使用编译而不是解释的原因。

随着java的到来,一种新的解释过程就被引入了。java语言能向任何计算机移植。为了取得可移植性,源程序到目标程序的翻译分成两步进行:编译和解释。java源程序首先被编译,创建java的字节代码,字节代码看起来像机器语言中的代码,但不是任何特定计算机的目标代码,它是一种虚拟机的目标代码,该虚拟机称为java虚拟机或JVM。字节代码然后能被任何运次JVM模拟器的计算机编译或解释,也就是运行字节代码的计算机只需要JVM模拟器,而不是java编译器。

编译和解释的不同在于,编译在执行前翻译整个源代码,而解释一次只翻译和执行源代码中的一行。但是,两种方法都遵循图中显示的相同的翻译过程。

词法分析器一个符号一个符号地读取源代码,创建源语言中的助记符表。

词法分析器分析一组助记符,找出指令。

语义分析器检查语法分析器创建的句子,确保它们不含有二义性。计算机语言通常是无二义性的,这意味着这一步骤或者是在翻译器中被省略,或者其责任被最小化。

在无二义性指令被语义分析器创建之后,每条指令将转化为一组程序将要在其上运行的计算机的机器语言。这个是由代码生成器完成的。

当今计算机语言按照它们使用的解决问题的方法分类。因此,模式是计算机语言看待要解决问题的一种方式。计算机语言课分成4中模式:过程性、面向对象、函数式和声说式。

在过程模式中我们把程序看成是操纵被动对象的主动主体。我们日常生活中遇到许多被动对象:石头、书、灯等。一个被动对象本身不能发出一个动作,但它能从主动主体接收动作。

过程式模式下的程序就是主动主体,该主体使用称为数据或数据项的被动对象。作为被动对象的数据项存储在计算机的内存中,程序操作他们。为了操纵数据,主动主体发出动作,称之为过程。

我们需要把过程和 程序触发区分开。程序不定义过程,它只触发或调用过程。过程必须已经存在。

当使用过程式高级语言时,程序仅由许多过程调用构成,除此之外没有任何东西。这不是显而易见的,但即使使用像加法运算符这样的简单数学运算符时,我们也是正在使用一个过程调用一个已经编写的过程。

如果我们考虑过程和被作用于的对象,那么过程式模式的概念就变得更为简单,而且容易理解。这种模式的程序由三部分构成:对象创建部分、一组过程调用和每个过程的一组代码。有些过程在语言本身中已经被定义。通过组合这些代码,开发者可以建立新的过程。

面向对象模式处理活动对象,而不是被动对象。我们在日常生活中遇到许多活动对象:汽车、自动门、洗盘机。在这些对象上执行的动作都包含在这些对象那中:对象只需要接收合适的外部刺激来执行其中一个动作。

总体上,方法的格式与有些过程式语言中用的函数非常相似。每个方法有它的头、局部变量和语句。这就意味着我们对过程式语言所讨论的大多数特性都可以应用在为面向对象程序所写的方法上。换言之,我们可以认为面向对象语言实际上是带有新的理念和新的特性的过程式语言的扩展。

在面向对象模式中,作为本质,一个对象能从另一个对象继承。这个概念被称为继承性。当一般类被定义后,我们可以定义继承了一般类中一些特性的更具体的类,同时这些类具有一些新特性。

多态性因为他们使用符号,所以这些语言首先被称为符号语言。这些助记符语言后来就被称为汇编语言。假想计算机用于替代机器语言的汇编语言显示在程序中。

尽管汇编语言大大提高了变成效率,但任然需要程序员在所使用的硬件上花费大部分精力。用符号语言编程也很枯燥,因为每条机器指令都必须单独编码。为了提高程序员效率以及从关注计算机转到关注解决的问题,促进了高级语言的发展。

高级语言可移植到许多不同的计算机,使得程序员能够将精力集中在应用程序上,而不是计算机结构的复杂性上。高级语言旨在使程序员摆脱汇编语言繁琐的细节。高级语言同汇编语言有一个共性:它们必须被转化为机器语言,这个转化过程被称为解释或编译。

数年来,人们开发了各种各样的语言,最著名的有BASIC、COBOL、Pascal、Ada、C、C++和Java。

当今程序通常是同一种高级语言来编写。为了在计算机上运行程序,程序需要被翻译成它要运行在其他的计算机的机器语言。高级语言程序被称为源程序。被翻译成的机器语言程序被称为目标程序。有两种方法用于翻译:编译和解释。

编译程序通常把整个源程序翻译成目标程序。

有些计算机语言使用解释器把源程序翻译成目标程序。解释是把源程序中的每一行翻译成目标程序中相应的行,并执行它的过程。但是,我们需要意识到在解释中的两种趋势:在java语言之前被有些程序使用的和java使用的解释。

在java语言之前的有些解释式语言使用一种被称为解释的第一种方法的解释过程,因为缺少其他任何名字,说以称为解释的第一种方法。在这种解释中,程序的每一行被翻译成其使用的计算机上的机器语言,该行机器语言被立即执行。如果在翻译或执行中有任何错误,过程就显示消息,其余的过程就被终止。程序需要被改正,再次从头解释和执行。第一种方法被看成是一种慢的过程,这就是大多数语言使用编译而不是解释的原因。

随着java的到来,一种新的解释过程就被引入了。java语言能向任何计算机移植。为了取得可移植性,源程序到目标程序的翻译分成两步进行:编译和解释。java源程序首先被编译,创建java的字节代码,字节代码看起来像机器语言中的代码,但不是任何特定计算机的目标代码,它是一种虚拟机的目标代码,该虚拟机称为java虚拟机或JVM。字节代码然后能被任何运次JVM模拟器的计算机编译或解释,也就是运行字节代码的计算机只需要JVM模拟器,而不是java编译器。

编译和解释的不同在于,编译在执行前翻译整个源代码,而解释一次只翻译和执行源代码中的一行。但是,两种方法都遵循图中显示的相同的翻译过程。

词法分析器一个符号一个符号地读取源代码,创建源语言中的助记符表。

词法分析器分析一组助记符,找出指令。

语义分析器检查语法分析器创建的句子,确保它们不含有二义性。计算机语言通常是无二义性的,这意味着这一步骤或者是在翻译器中被省略,或者其责任被最小化。

在无二义性指令被语义分析器创建之后,每条指令将转化为一组程序将要在其上运行的计算机的机器语言。这个是由代码生成器完成的。

当今计算机语言按照它们使用的解决问题的方法分类。因此,模式是计算机语言看待要解决问题的一种方式。计算机语言课分成4中模式:过程性、面向对象、函数式和声说式。

在过程模式中我们把程序看成是操纵被动对象的主动主体。我们日常生活中遇到许多被动对象:石头、书、灯等。一个被动对象本身不能发出一个动作,但它能从主动主体接收动作。

过程式模式下的程序就是主动主体,该主体使用称为数据或数据项的被动对象。作为被动对象的数据项存储在计算机的内存中,程序操作他们。为了操纵数据,主动主体发出动作,称之为过程。

我们需要把过程和 程序触发区分开。程序不定义过程,它只触发或调用过程。过程必须已经存在。

当使用过程式高级语言时,程序仅由许多过程调用构成,除此之外没有任何东西。这不是显而易见的,但即使使用像加法运算符这样的简单数学运算符时,我们也是正在使用一个过程调用一个已经编写的过程。

如果我们考虑过程和被作用于的对象,那么过程式模式的概念就变得更为简单,而且容易理解。这种模式的程序由三部分构成:对象创建部分、一组过程调用和每个过程的一组代码。有些过程在语言本身中已经被定义。通过组合这些代码,开发者可以建立新的过程。

面向对象模式处理活动对象,而不是被动对象。我们在日常生活中遇到许多活动对象:汽车、自动门、洗盘机。在这些对象上执行的动作都包含在这些对象那中:对象只需要接收合适的外部刺激来执行其中一个动作。

总体上,方法的格式与有些过程式语言中用的函数非常相似。每个方法有它的头、局部变量和语句。这就意味着我们对过程式语言所讨论的大多数特性都可以应用在为面向对象程序所写的方法上。换言之,我们可以认为面向对象语言实际上是带有新的理念和新的特性的过程式语言的扩展。

在面向对象模式中,作为本质,一个对象能从另一个对象继承。这个概念被称为继承性。当一般类被定义后,我们可以定义继承了一般类中一些特性的更具体的类,同时这些类具有一些新特性。

多态性.

人们已经发明了一些面向对象语言。我们简要讨论其中两种语言的特性:C++和java。

Java是由Sun Microsystems公司开发的,它在C和C++的基础上发展而来,但是C++的一些特性从语言中被移除,从而使java更健壮。另外,改语言是完全面向类操作的。在C++中,你甚至可以不用定义类就能解决问题。而在java中,每个数据项都属于一个类。

java中的程序可以是一个应用程序也可以是一个小程序。应用程序是指一个可以完全对运行的程序。小程序则是嵌入超文本标记语言中的程序,存储在服务器上并由浏览器运行。浏览器也可以把它从服务器下载到本地运行。

在java中,应用程序(或小程序)是类以及类实例的集合。java自带的丰富类库是它的有趣特征之一。尽管C++也提供类库,但是java中用户可以在提供的类库基础上构建新类。

在java中,程序的执行也是独具特色的。构建一个类并把它传给编译器。由编译器来调用类的方法。java的另一大有趣的特点是多线程。线程是指按顺序执行的动作序列。C++只允许单线程执行,但是java允许多线程执行。

在这一节,我们通过对一些过程式语言的快速浏览,发现共同的概念。这些概念中的一些读大多数面向对象也适用,这是因为当创建方法时面向对象模式使用过程式模式。

所有计算机语言的共同特点之一就是具有标识符,即对象的名称。标识符允许给程序对象命名。

数据类型定义了一系列值及应用于这些值的一系列操作。每种数据类型值的集合被称为数据类型的域。大多数语言都定义了两类数据类型:简单数据类型和复合数据类型。

简单数据类型是不能分解成更小数据类型的数据类型。

复合数据类型是一组元素,其中每个元素都是简单数据类型或复合数据类型。

变量是存储单元的名字。

字面值是程序中使用的预定义的值。

常量

输入和输出

表达式是由一系列操作数和运算符简化后的一个单一数值。运算符。操作数。

每条语句都使程序执行一个相应动作。它被直接翻译成一条或多条计算机可执行指令。
