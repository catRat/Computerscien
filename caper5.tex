
本章我们将讨论计算机的组成。讲解计算机是如何由三个子系统组成的。我们还介绍了简单假想的计算机,它能运行简单程序,完成基本的算术或逻辑运算。

计算机的组成部件可以分为三大类:中央处理单元、主存储器和输入/输出子系统。接下来的三个部分将讨论这些子系统以及如何将这些子系统组成一台计算机。

中央处理器用于数据的运算。在大多数体系结构中,它有三个组成部分:算术逻辑单元、控制单元、寄存器组、快速存储定位。

算术逻辑单元对数据进行逻辑、位移和算术运算

在第4章中,我们讨论了几种逻辑运算,如:非、与、或和抑或。这些运算大输入数据作为二进制位模式,运算的结果也是二进制模式。

在第4张中,我们讨论了数据的两种位移运算:逻辑移位和算术移位运算。逻辑移位运算用来对二进制位模式进行向左或右的移位,而算术运算被应用于整数。它们的主要用途是用2除或乘一个整数。

寄存器是用来存放临时数据的高速独立的存储单元。CPU的运算离不开大量寄存器的使用。

在过去,计算机只有几个数据寄存器用来存储输入数据和运算结果。现在,由于越来越多的复杂运算改由硬件设备实现,所以计算机在CPU中使用几十个寄存器来提高运算速度,并且需要一些寄存器来保存这些运算的中间结果。

现在,计算机存储的不仅是数据,还有存储在内存中相对应的程序。CPU的主要职责是:从内存中逐条取出指令,并将取出的指令存储在指令寄存器中,解释并执行指令。

CPU中另一个通用寄存器是程序计数器。程序计数器中保存这当前正在执行的指令。当前的指令执行完后,计数器敬将自动加1,指向下一条指令的内存地址。

CPU的第三个部分是控制单元,控制单元控制各个子系统的操作。控制是通过从控制单元到其他子系统的信号进行的。

主存储器是计算机内的第二个子系统。它是存储单元的集合,每一个存储单元都有唯一的标识,称为地址。数据以成为字的位组形式在内存中传入和传出。字可以是8位、16位、32位,甚至有时是64位,如果字是8位,一般称为1字节。术语字节在计算机科学中使用相当普遍,因此有时成为16位位2字节,32位位4字节。

在存储器中存取每个字都需要有相应的标识符。尽管程序员使用命名的方式来区分,但在硬件层次上,每个子都是通过地址来标识的。所以在内存中表示的独立的地址单元的总数称为地址空间。

随机存取存储器是计算机中主存的主要组成部分。在随即存取设备中,可以使用存储单元地址来随机存取一个数据项,而不需要存取位于它前面的所有数据项。该术语有时因ROM也能随机存取而与ROM混淆,

计算机用户需要许多存储器,尤其是速度快而且价格低廉的存储器。但这种要求并不总能得到满足。存取数度快的存储器通常都不便宜。因此需要寻找一种折中的方法。解决的办法是采用存储器的层次结构。
