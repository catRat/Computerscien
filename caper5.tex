\chapter{计算机组成}
本章我们将讨论计算机的组成。讲解计算机是如何由三个子系统组成的。我们还介绍了简单假想的计算机,它能运行简单程序,完成基本的算术或逻辑运算。

\section{引言}
计算机的组成部件可以分为三大类:中央处理单元、主存储器和输入/输出子系统。接下来的三个部分将讨论这些子系统以及如何将这些子系统组成一台计算机。
\section{中央处理器}
中央处理器用于数据的运算。在大多数体系结构中,它有三个组成部分:算术逻辑单元、控制单元、寄存器组、快速存储定位。
\subsection{算术逻辑单元}
算术逻辑单元对数据进行逻辑、位移和算术运算
\begin{enumerate}
	\item 逻辑运算

	在第4章中,我们讨论了几种逻辑运算,如:非、与、或和抑或。这些运算大输入数据作为二进制位模式,运算的结果也是二进制模式。
	\item 位移运算

	在第4张中,我们讨论了数据的两种位移运算:逻辑移位和算术移位运算。逻辑移位运算用来对二进制位模式进行向左或右的移位,而算术运算被应用于整数。它们的主要用途是用2除或乘一个整数。
	\item 算术运算

	在第4章我们讨论了整数和实数上地一些算术运算,我们提到有些运算能被更高效地硬件实现。
\end{enumerate}
\subsection{寄存器}
寄存器是用来存放临时数据的高速独立的存储单元。CPU的运算离不开大量寄存器的使用。
\begin{enumerate}
	\item 数据寄存器

	在过去,计算机只有几个数据寄存器用来存储输入数据和运算结果。现在,由于越来越多的复杂运算改由硬件设备实现,所以计算机在CPU中使用几十个寄存器来提高运算速度,并且需要一些寄存器来保存这些运算的中间结果。
	\item 指令寄存器 

	现在,计算机存储的不仅是数据,还有存储在内存中相对应的程序。CPU的主要职责是:从内存中逐条取出指令,并将取出的指令存储在指令寄存器中,解释并执行指令。

	\item 程序计数器

	CPU中另一个通用寄存器是程序计数器。程序计数器中保存这当前正在执行的指令。当前的指令执行完后,计数器敬将自动加1,指向下一条指令的内存地址。
\end{enumerate}
\subsection{控制单元}
CPU的第三个部分是控制单元,控制单元控制各个子系统的操作。控制是通过从控制单元到其他子系统的信号进行的。
\section{主存储器}
主存储器是计算机内的第二个子系统。它是存储单元的集合,每一个存储单元都有唯一的标识,称为地址。数据以成为字的位组形式在内存中传入和传出。字可以是8位、16位、32位,甚至有时是64位,如果字是8位,一般称为1字节。术语字节在计算机科学中使用相当普遍,因此有时成为16位位2字节,32位位4字节。
\subsection{地址空间}
在存储器中存取每个字都需要有相应的标识符。尽管程序员使用命名的方式来区分,但在硬件层次上,每个子都是通过地址来标识的。所以在内存中表示的独立的地址单元的总数称为地址空间。
\subsection{存储器的类型}
主要由两种类型的存储器:RAM和ROM。
\begin{enumerate}
	\item RAM

	随机存储器(RAM)是计算机中存储器的主要组成部分。在随机存储设备中,可以使用存储单元地址来随机存储一个数据项,而不需要存取位于它前面的所有数据项。该术语有时因ROM也能随机存取而与ROM混淆,用户可读性RAM,即用户可以在RAM中写信息,之后可以方便地通过覆盖来查除原有信息。RAM的另一个特点是易失性。当系统断电后,信息将丢失。换句话说,当计算机断电后,存储在RAM中的信息将被删除。RAM技术又可分为两大类:SRAM和DRAM。
	\begin{enumerate}
		\item SRAM

		静态RAM(SRAM)技术是用传统的触发器门电路来保存出局。这些门保持状态(0或1),也就是说当通电的时候数据始终存在,不需要刷新。SRAM速度快,但是价格昂贵。
		\item DRAM

		动态RAM(DRAM)技术使用电容器。如果电容器充电,则这时的状态是1;如果放电则状态时0.因为电容器会随时间而漏掉一部分电,所以内存单元需要周期性地刷新。DRAM比较慢,但是价格便宜。
	\end{enumerate}
	\item ROM

	只读存储器(ROM)的内容时由制造商写进去的。用户只能读但不能写,它的优点时非易失性:当切断电源后,数据不会丢失。通常用来存储那些关机后也不能丢失的程序或数据。例如,用ROM来存储那些在开机时运行的程序。
	\item PROM

	称为可编程只读存储器(EPROM)的一种ROM。这种存储器在计算机发货时是空白的。计算机用户借助一些特殊的设备可以将程序存储在上面。当程序被存储后,它就会像ROM一样不能重写。也就是说,计算机使用者可以用它来存储一些特定的程序。
	\item EEPROM

	称为电可擦除可编程只读存储器(EEPROM)的一种EPROM。对它的编程和擦除用电子脉冲即可,无须从计算机上拆下来。

\end{enumerate}

\subsection{存储器的层次结构}
计算机用户需要许多存储器,尤其是速度快而且价格低廉的存储器。但这种要求并不总能得到满足。存取数度快的存储器通常都不便宜。因此需要寻找一种折中的方法。解决的办法是采用存储器的层次结构。该层次结构如下:
\begin{enumerate}
	\item 当对速度要求苛刻时可以使用少量告诉存储器。CPU中的寄存器就是这种存储器。
	\item 用适量的中速存储器来存储经常需要访问的数据。例如下面将要讨论的告诉缓冲存储器就属于这一类。
	\item 用大量的低俗存储器存储那些不经常访问的数据。
\end{enumerate}
\subsection{高速缓冲存储器}
高速缓冲存储器的存取速度要比主存快,但是比CPU及其内部的寄存器要慢。高速缓冲存储器通常容量较小,且常被置于CPU和主存之间。

高速缓冲存储器在任何使劲按都含有主存的一部分内容的副本。当CPU要存取主存中的一个字的时候,将按一下步骤进行:
\begin{enumerate}
	\item CPU首先检查高速缓冲存储器。
	\item 如果要存取的字存在,CPU就将它复制;如果不存在,CPU将从主存中复制一份从需要读取的字开始的数据块。该数据块将覆盖高速缓冲存储器中的内容。
	\item CPU存取高速缓冲存储器并赋值该字。
\end{enumerate}
这种方式将提高运算的速度;如果字在高速缓冲存储器中,就立即存取它。如果字不再高速缓冲存储器中,字和整个数据块就会被复制到高速缓冲存储器中。因为很有可能CPU在下次存取红需要存取从上次存取的第一个字后续的字,所以高速缓冲存储器可以大大提高处理器的速度。

读者可能会奇怪为什么高速存储器尽管存储容量小效率却很高,这是由于80-20规则。据观察,通常计算机花费80\%的时间来读取20\%的数据。换句话说,相同的数据往往被存储多次。高速缓冲存储器,凭借其高速,可以存储这20\%的数据而使存取至少快80\%。
\section{输入/输出子系统}
计算机的第三个子系统是称为输入/输出(I/O)子系统的一系列设备。这个子系统可以使计算机与外部通信,并在断电的情况下存储程序和数据。输入/输出设备可以分为两大类:非存储设备和存储设备。
\subsection{非存储设备}
非存储设备使得CPU/内存可以与外界通信,但它们不能存储信息。
\begin{enumerate}
	\item 键盘和监视器

	两个最常见的非存储输入/输出设备使键盘和监视器。键盘提供输入功能;监视器显示输出并同时响应键盘的输入。程序、命令和数据的输入或输出都是通过字符串进行的。字符则是通过字符集(如ASSII码)进行编码。此类中其他的设备由鼠标、操纵杆等。
	\item 打印机

	打印机是一种用于产生永久性记录的输出设备。它是非存储设备,因为要打印的材料不能够直接由打印机输入计算机中,而且也不饿能再次利用,除非有人通过打字或扫描的方式再次输入计算机中。
\end{enumerate}
\subsection{存储设备}
尽管存储设备被分为输入/输出设备,但它可以存储大量的信息以备后用。它们要比主存便宜得多,而且存储的信息也不易丢失(即使断电信息也不会丢失)。有时称它们为辅助存储设备,通常分为磁介质和光介质两种。
\begin{enumerate}
	\item 磁介质存储设备

	磁介质存储设备使用磁性来存储数据。如果一点有磁性则表示1,如果没有磁性则表示0.
	\begin{itemize}
		\item 磁盘

		磁盘是由一张一张的磁片叠加而成的。这些此片由薄磁膜封装起来。i洗脑洗是通过盘上每一个磁片的读/写磁头写磁介质表面来进行读取和存储的。
		\item 磁带

		磁带的大小不一。通常的一种是用厚磁膜封装的半英寸塑料磁带。
	\end{itemize}
	\item 光存储设备

	光存储设备是一种新技术,它使用光技术来存储和读取数据。在发明了CD后人们利用光存储技术来保持音频信息。现在,相同的技术(稍作改进)被用于存储计算机上的信息。使用这种技术的设备有只读光盘(CD-ROM)、可刻录光盘(CD-R)、可重写光盘(CD-RW)、数字多功能光盘(DVD)。
	\begin{itemize}
		\item CD-ROM

		只读光盘使用于CD相同的技术(该技术最初是由飞利浦和索尼公司为录制音乐而研发的)。两者间位移的区别在于增强程度不同;CD-ROM更健壮,而且纠错能力较强。
	\end{itemize}
\end{enumerate}
\section{子系统的互连}
前面的几节中已经介绍了在单个计算机上的三个子系统的主要特点。本节将介绍它们三者之间在内部是如何连接的,内部连接扮演很重要的角色,因为i洗脑洗需要在这三个子系统中交换。
\subsection{CPU和存储器的连接}
CPU和存储器之间通常由称为总线的三组线路连接在一起,它们分别是:数据总线、地址总线和控制总线。
\subsection{输入/输出设备的寻址}
通常CPU使用相同的总线在主存和输入/输出设备之间读写数据。
\section{程序执行}
当今,通用计算机使用称为程序的一系列指令来处理数据。计算机通过执行程序,将输入数据转换成输出数据。程序和数据都放在内存中。

本章最后将给出假象简单计算机执行程序的例子。
\subsection{机器周期}
\subsection{输入/输出操作}
\section{不同的体系结构}
