
在本章,我们讨论数据库和数据库管理系统(DBMS)。我们给出数据库管理系统的三层结构。重点讲解关系数据库模型并举例说明其运算。接着,介绍一种在关系数据库上使用的语言。最后,简要介绍数据库设计和其他的数据库模型。\par
数据的存储传统上是使用单独的没有关联的文件,有时称为平面文件。在过去,组织中的每个应用程序都使用自己的文件。例如,在一所大学中,每个部门可能会有他们自己的文件集合,成绩记录办公室保存了关于学生信息和他们成绩的文件;经济资助办公室保存了他们自己的关于需要经济资助完成学业的学生的文件;调度办公室保存了教授的姓名和他们所教课程;工资部门保存了他们自己的关于全体教职工的文件,等等。但是,现在所有这些平面文件被组合成一个实体--一个全大学的数据库。\par
虽然要给出一个广泛接受的数据库定义有一些困难,但我们使用下面通常的定义:\par
数据库是一个组织内被应用程序使用的逻辑相一致的相关数据的集合。\par
与平面文件系统相比,我们可以说出数据库系统的几个优点:\par
平面文件系统中存在大量的冗余。\par
如果相同的信息被存储在多个地方,那么对数据的任何修改需要在数据存储的所有地方进行。例如,一个女学生结婚了,并接受了她丈夫的姓,那么这个学生的姓需要在所有包含该学生信息的地方做修改。一不小心很容易造成数据的不一致性。\par
数据库通常比平面文件系统的效率要高得多,因为数据库中一条信息存储在更少的地方。\par
数据库系统更容易维护数据的完整性,因为数据信息存储在更少的地方。\par
如果数据是集中存放在一个地方,这就更容易维护信息的机密性。\par
数据库管理系统(DBMS)是定义、创建、维护数据库的一种工具。DBMS也允许用户来控制数据库中的数据的存取。数据库管理系统由5部分构成:硬件、软件、数据、用户和规程。\par
硬件是指允许物理上存取数据的计算机硬件系统。例如,用户终端、硬盘、主机和工作站,都被认为是DBMS的硬件组成部分。\par
软件是指允许用户存取、维护和更新物理数据的实际程序。另外,软件工具还可以控制哪些用户可以对数据库中的哪部分数据进行存取。\par
数据库中的数据存储在物理存储设备上。数据是独立于软件的一个实体。这种独立使得组织可以在不改变物理数据及其存取方式的情况下,更换所应用的软件。如果组织决定使用数据库管理系统,那么该组织所需要的所有信息必须存放在一个实体中,从而便于DBMS中的软件存取。\par
术语用户在数据库管理系统中有广泛的定义。我们可以将用户分为两类:组中用户和应用程序。\par
最终用户指直接从数据库中获取信息的用户。最终用户又可分为两类:数据库管理员(DBA)和普通用户。数据库管理员拥有最大的权限,可以控制其他用户以及他们对数据库的存取。数据库管理员可以将他的一些特权授予其他用户并保留随时收回特权的能力。而另一方面,普通用户只能呢使用部分数据库和有限的存取。\par
数据库中的数据的其他使用就是应用程序。应用程序需要存取的处理数据。例如,工资单应用程序就要存取数据库中的数据来生产月底的工资单。\par
数据库管理系统的最后一个部分就是必须被明确定义并为数据库用户所遵循的规程或规则的集合。\par
美国国家标准协会/标准计划和需求委员会(ANSI/SPARC)为数据库管理系统建立了三层体系结构:内层、概念层和外层。\par
内层决定了数据在存储设备中的实际位置。这个层次处理低层次的数据存取方法和如何在存储设备键传输字节。换句话说,内层直接和硬件交互。\par
概念层定义数据的逻辑视图。在该层中定义了数据模式。数据库管理系统的主要功能都在该层。数据库管理系统把数据内部视图转化为用户所蛋刀的外部视图。概念层是中介层,它使得用户不必与内层打交道。\par
外层直接与用户交互。它将来自概念层的数据转化为用户所熟悉的格式和视图。\par
数据库模型定义了数据的逻辑设计,它也描述了不同数据之间的联系。在数据库设计发展史中,曾使用过三种数据库模型:层次模型、网状模型和关系模型。\par
层次模型中,数据被组织成一棵倒置的树。\par
在关系数据库管理系统(ROBMS)中,数据通过关系的集合来表示的。\par
从表面上看,关系就是二维表。在关系数据库管理系统中,数据的外部视图就是关系或表的集合,但这并不表示数据以表的形式存储。数据的物理存储于数据的逻辑组织的方式毫无关系。\par
关系数据库管理系统中的关系有以下的特征:\par
在关系数据库管理系统中,每一张关系具有唯一的名称。\par
在关系中的每一列都称为属性,属性在表中是列的头。么一个属性表示了存储在该列下的数据的含义。表中的每一列在关系范围内有唯一的名称。关系中属性的总数称为关系的度。\par
关系中的行交元祖。元祖定义了一组属性值。关系中元祖的个数叫作关系的基数。当增加或减少元祖时,关系的基数就会改变。这就实现了动态数据库。\par
在关系数据库中,我们可以定义一些操作来通过抑制的关系创建新的关系。本节中共定义了9种操作:插入、删除、更新、选择、投影、连接、并、交和差。我们并不抽象地讨论这些操作,而是把它们描述成在数据库查询语言SQL(结构化查询语言)中的定义。 \par
结构化查询语言(SQL)是美国国家协会(ANSI)和国际标准组织(ISO)用于关系数据库上的标准化语言。这是一种描述性的语言,这意味着使用者不需要一步步地编写详细地程序而只需声明它。结构化查询语言与1979年首次被Oracle公司是实现。只有有了更多的新版本。
\begin{verbatim}
insert into RELATION-NAME values (...,...,...)
delete from RELATION-NAME where criteria
update RELATION-NAME
set attributel=valuel, attribute2=value2, ...
where criteria
select * from RELATION-NAME where criteria
select attribute-list from RELATION-NAME
select attribute-list from RELATION1, RELATION2 where criteria
select * from RELATION1 union select * from RELATION2
select * from RELATION1 intersection select * from RELATION2
select * from RELATION1 minus select * from RELATION2
\end{verbatim}
数据库的设计是一个冗长且只能通过一步步过程来完成的任务。第一步通常涉及与数据库潜在用户的面谈,去收集需要存储的信息和每个部门的存取需求。第二步就是建立一个实体关系模型(ERM),这种模型定义了其一些信息需要维护的实体、这些实体的属性和实体间的关系。 \par
设计的下一步是基于使用的数据库类型的。在关系数据库中,下一步就是建立基于ERM的关系和规范化这些关系。在这门介绍性的课程中,我们仅仅给出ERM和规范化的一些概念。 \par
在这一步,数据库设计者建立了实体关系(E-R)图来表示那些其信息需要保存的实体和实体间的关系。E-R图使用了多种几何图形,但这里只使用其中的少部分: \par
在E-R图完成后,关系数据库中的关系就建立了。\par
规范化是一个处理过程,通过此过程给定一组关系转化成一组具有更坚固结构的新关系。规范化要允许数据库中表示的任何关系,要允许像SQL这样的语言去使用由原子操作组成的恢复操作,要移除插入、删除和跟新操作中的不规则,要减少当新的数据类型被加入时对数据库重建的需要。\par
规范化过程定义了一组层次范式(NF)。多种范式已经被提出,包括1NF、2NF、3NF、BCNF、4NF、PJNF和5NF等。这些范式的讨论涉及函数依赖性的讨论,这是一门理论的学科,超出了本书的范围,这里我们只是从兴趣出发简单地讨论其中的一些。但是,有一点要知道,那就是这些范式形成了一个层次结构。
