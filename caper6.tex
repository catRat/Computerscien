\chapter{计算机网络和因特网}
个人计算机的发展带动了商业、工业、科学和教育的巨大改变。网络也发生了类似的变革。技术的进步使得通信线路能传送更多、更快的信号。而不断发展的服务使得我们能够使用这些扩展的能力。计算机网络的研究导致了新技术的产生——在全球各个地方交换文本、音频和视频等数据,在任何时候快速、准确地下载或上载信息。
\section{引言}
虽然本章的目标是讨论因特网,一个将世界上几十亿台计算机互相连接的系统,我们对互联因特网的认识不应该是一个单独的网络,而是一个网络结合体,一个互联网络。因此,我们的旅程将从定义一个网络开始。然后我们将要展示如何通过网络连接来建造小型的互联网络。最终我们会展示因特网的结构并且在本章的以下部分开启研究因特网的大门。
\subsection{网络}
网络是一系列可用于通信的设备相互连接构成的。在这个定义里面,一个设备可以是一台主机,比如一台大型计算机、台式机、便携式计算机、工作站、手机或安全系统。在这种定义中,设备也可以是一个连接设备,比如用来将一个网络与另一个网络相连接的路由器,一个将不同设备连接在一起的交换机,或者一个用于改变数据形式的调制器,等等。在一个网络中,这些设备都通过有线或无线传输媒介互相连接。当我们在家通过即插即用路由器连接两台计算机时,虽然规模很小,但已经建造了一个网络。
\begin{enumerate}
	\item 局域网

	局域网通常是与单个办公室、建筑或校园内的几个主机相连的私有网络。基于机构的需求,一个局域网既可以简单到某人家庭办公室中的两台个人计算机和一台打印机,也可以扩大至一个公司范围,并包括音频和视频设备。在一个局域网中的每一台主机都作为这台主机在局域网中唯一定义的一个标识和一个地址。一台主机向另一台主机发送的数据包中包括源主机和目标主机的地址。

	\item 广域网

	广域网也是通信设备互相连接构成的。但是广域网与局域网之间有一些差别。局域网的大小通常是受限的,跨越一个办公室、一座大楼或一个校园;而广域网的地理跨度更大,可以跨越一个城镇、一个州、一个国家,甚至横跨世界。局域网将主机互联,广域网则将交换机、路由器或调制解调器之类的连接设备互连。通常,局域网为机构私有,广域网则由通信公司创建并运营,并且租给使用它的机构。我们可以看到广域网的两种截然不同的案例:点对点广域网和交换广域网。

	点对点广域网是通过传输媒介连接两个通信设备的网络。

	交换广域网是一个有至少两个端的网络。就像我们很快就会看到的那样,交换广域网用于今天全球通信的骨干网。我们也可以这么说,交换广域网是一个点对点广域网通过开关连接产生的结合体。

	\item 互联网络

	现在很难看见独立存在的局域网或广域网,它们现在都是互相连接的。当两个或多个网络相互连接时,它们构成一个互联网络,或者说网际网。
\end{enumerate}
\subsection{因特网}
正如我们之前讨论过的,一个网际网是两个或多个可以互相通信的网络。最值得注意的网际网是因特网,它由成千上万个互连的网络组成。图展示了因特网的一个概念图像。
\subsection{硬件和软件}

我们已经展示了因特网通过连接设备将大大小小的网络相互交织在一起构成的基本结构。然而,如果仅仅将这些部分连接在一起,很明显什么都不会发生。为了产生沟通,即需要硬件也需要软件设备。这就像当进行一个复杂的计算时,我们同时需要计算机和程序。

\subsection{协议分层}

当讨论因特网时,有一个词我们总是会听见,这个词就是协议。协议定义了发送器、接收器以及所有中间设备必须遵守以保证有效地通信的规则。简单的通信可能只需要一条简单的协议,当通信便得复杂时,可能需要将任务分配到不同的协议层中,在这种情况下,我们在每一个协议层都需要一个协议,或者协议分层。

\begin{enumerate}
	\item 情景

	为了更好地理解协议分层的必要,首先我们开发一个简单的场景。假定Ann和Maria是有很多共同想法的邻居,她们每次都会为了一个何时退休的计划互相见面和进行沟通。突然,Ann所在的公司为她提供一个升职的机会,但是同时要求她搬到离Maria很远的一个城市中的分部去住。由于她们想出了一个具有创新型的计划——在退休后开始做新的生意,这两个朋友仍然向继续它们的通信并且就这个新计划交换想法。她们决定通过到邮局使用普通邮件通信来继续她们的对话,但是如果邮件被拦截了,她们不想别人知晓它们的想法。她们在邮件的加密/解密方式上达成了一致,寄信人对邮件进行加密,这样对于入侵者而言邮件是无法阅读的;同时收件人对邮件进行解密以得到原始邮件。第16章将讨论加密/解密方法,但是现在我们可以推测Maria和Ann使用了其中一种使没有密钥的人很难那对邮件进行解密的技术。这样,可以把Maria和Ann之间的通信分为三个协议层,如图所示。假设Ann和Maria都各自拥有三台机器(或机器人)来完成每一个协议层的任务。

	让我们假设Maria将第一封邮件寄给Ann。Maria假设第三协议层机器是正在听她说话的Ann并对其说话,第三层协议层的机器听她说话并且创作出明文,并被传到第二层的机器,第二层的机器对文本进行加密,将它创作成密文,并传送到第一协议层的机器。这个第一协议层的机器,也许是一个机器人,把密文装进信封,加上发信人地址,然后将信寄出。

	在Ann这边,第一协议层的机器从Ann的邮件中取出邮件并通过发信人地址找出来自Maria的这一封。它从信封中取出密文并传递给第二协议层的机器,第二协议层的机器对密文进行解密,创作出明文并传递给第三协议层的机器,第三协议层的机器接受明文并且将它读出来,就像Maria在说话一样。

	协议分层使我们将大任务化简成几个更小、更简单的任务。例如,在上述中,我们可以只用一台机器来完成三台机器全部的工作,但是,如果Maria和Ann觉得机器完成的现有加密或解密无法保证她们保密,他们呢需要替换整个机器。她现在的情况下,她们只需要替换第二协议层的机器就足够了,另外两层可以保持不变;这称为模块化。在这里模块化指的是独立的协议层。一个协议层(模块)可以定义位一个具有输入和输出而不需要考虑输入时如何变成输出的黑匣子。当乡两台机器提供相同输入得到相同输出时,它们就可以相互替换。

	协议分层的一个优势就是可以将服务和其实施分开来。每层使用更底层的服务,并向较高一层提供服务;并且我们不需要考虑该层时如何实施的。例如,Maria也许可以自己完成第一协议层的工作而决定不买第一协议层的计算机(机器人)。只要Maria可以完成第一协议层提供的工作,这个通信系统就可以正常的双向运行。

	协议层的另一个优势,虽然可在简单的例子中无法看出,但是在我们讨论因特网中的协议分层时会表现出来,因为通信系统往往不仅仅具有两个端系统,还有一些只需要几个协议层而不是所有协议层的中间系统。如果我们不使用协议分层,整个系统会变得更复杂,因为那样我们得把每一个中间系统都变得和端系统一样复杂。

	协议分层有什么劣势吗?也许有人会认为单一协议层可以是整个工作变简单,而且没有每个协议层都使用低一级协议层的服务并向高一级协议层提供服务的必要。但是,就像之前提到的那样,一旦密码被破解,她们每一人得将整个机器换成新的而不是仅仅更换第二协议层。
	\item 协议分层的原则

	我们讨论一下分层的原则。第一原则规定,如果我们想达到双向通信,我们需要保证每一个协议层都可以进行两个对立且方向相反的工作。

	\item 逻辑连接

	在遵循以上两条原则之后,我们可以如图所示理解每个协议层之间的逻辑连接。这说明了层与层之间通信的存在。Maria和Ann可以认为她们能够发送该协议层创作的对象是基于每层的逻辑(假想的)连接。逻辑连接的概念会帮助我们更好地理解在数据通信和建立数据关系的网络中遇到的分层工作。

\end{enumerate}
\subsection{TCP/IP协议族}
通过第二个场景我们了解了协议分层和协议层之间的逻辑通信,这样就可以介绍传输控制协议/网际协议(TCP/IP)。如今因特网中使用的协议集(一组通过不同分层进行组织的协议)被称为TCP/IP协议族。TCP/IP协议族是一个分层协议,它由提供特定功能的交互式模块组成。分层这个术语说明每一个高层协议都基于一个或多个低层协议提供的服务。TCP/IP协议族被定义成图所示布局的软件层。
\begin{enumerate}
	\item 分层架构

	为了展示TCP/IP协议族中的分层是如何在两台主机通信中作用的,我们假设要使用的一套痛惜是在一个由3个LAN(连接)构成的小网络,且链路层开关与每个LAN相连。同时我们假定这些链接都与同一个路由器相连。

	\item 地址和数据包名称

	在因特网中,另外两个和协议分层有关的概念很值得一提,它们是地址和数据包名称。就像我们在之前讨论过的那样,在这个模型中,我们由层组之间的逻辑通信。任何涉及两步 校验的通信需要两个地址:源地址和目标地址。虽然看上去我们需要5组地址,每个协议层一组,但是正常情况下我们只有4组,因为物理层不需要地址。这是由于物理层数据交换的单位是位,这使它无法得到地址。
\end{enumerate}

	在应用层,我们通常使用名称来第一提供服务的站点,比如someorg.com,或者邮箱地址,比如somebody@coldmail.com。在传输层,地址被称为端口号,这些端口号的作用是在源和目的之间定义应用层程序。端口号的作用是通过各程序的本地地址来辨别多个同时运行的本地程序。在网络层,这些地址在整个因特网范围下是全球化的,网络层的地址独一无二得顶一个了该设备与因特网的连接。链路层地址,有时称为MAC地址,是在本地定义的地址,每一个链路层地址在计算机网络或广域网中定义一个特定的主机或者路由器。
\section{应用层}
在简单讨论网络、互联网和因特网之后,我们准备好了来进行一些关于TCP/IP协议族中每一层的讨论。我们从第5层像第1层讨论。

TCP/IP协议的第5层叫做应用层。应用层像用户提供服务。通信由逻辑连接提供,也就是说,假设两个应用层通过之间假想的直接连接发送和接受消息。

\subsection{提供服务}
应用层与其他层不同的地方在于,它是协议族中的最高层。在这层中的协议不向其他协议提供服务,它们只接收在传输层的协议提供的服务。这意味着该层的协议可以轻易去除。只要新的协议可以使用传输层中任意一个协议提供的服务,这个新的协议就可以添加到应用层上。

\subsection{应用层模式}

在网络的生命周期中,应用程序发展出了两种模式来解答这个问题:客户机-服务器模式和端到端模式。这里简单介绍这两种模式。
