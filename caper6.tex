个人计算机的发展带动了商业、工业、科学和教育的巨大改变。网络也发生了类型的变革。技术的进步使得通信线路能传送更多、更快的信号。而不断发展的服务使得我们嫩果使用这些扩展的能力。计算机网络的研究导致了新技术的产生——在全球各个地方交换文本、音频和视频等数据,在任何时候快速、准确地下载或上载信息。

虽然本章的目标是讨论因特网,一个将世界上几十亿台计算机互相连接的系统,我们对互联因特网的认识不应该是一个单独的网络,而是一个网络结合体,一个互联网络。因此,我们的旅程将从定义一个网络开始。然后我们将要展示如何通过网络连接来建造小型的互联网络。最终我们会展示因特网的结构并且在本章的与下部分开启研究因特网的大门。

网络是一系列可用于通信的设备相互连接构成的。在这个定义里面,一个设备可以是一台主机,比如一台大型计算机、台式机、便携式计算机、工作站、手机或安全系统。在这种定义中,设备也可以是一个连接设备,比如用来将一个网络与另一个网络相连接的路由器,一个将不同设备连接在一起的交换机,或者一个用于改变数据形式的调制器,等等。在一个网络中,这些设备都通过有线或无线传输媒介互相连接。当我们在家通过即插即用路由器连接两台计算机时,虽然规模很小,但已经建造了一个网络。

局域网通常是与单个办公室、建筑或校园内的几个主机相连的私有网络。基于机构的需求,一个局域网既可以简单到某人家庭办公室中的两台个人计算机和一台打印机,也可以扩大至一个公司范围,并包括音频和视频设备。在一个局域网中的每一台主机都作为这台主机在局域网中唯一定义的一个标识和一个地址。一台主机向另一台主机发送的数据包中包括源主机和目标主机的地址。

广域网也是通信设备互相连接构成的。但是广域网与局域网之间有一些差别。局域网的大小通常是受限的,跨越一个办公室、一座大楼或一个校园;而广域网的地理跨度更大,可以跨越一个城镇、一个州、一个国家,甚至横跨世界。局域网将主机互联,广域网则将交换机、路由器或调制解调器之类的连接设备互连。通常,局域网为集后私有,广域网则由通信公司创建并运营,并且租给使用它的机构。我们可以看到广域网的两种截然不同的案例:点对点广域网和交换广域网。

点对点广域网是通过传输媒介连接两个通信设备的网络。

交换广域网是一个有至少两个端的网络。就像我们很快就会看到的那样,交换广域网用于今天全球通信的骨干网。我们也可以这么说,交换广域网是一个点对点广域网通过开关连接产生的结合体。

现在很难看见独立存在的局域网或广域网,它们现在都是互相连接的。当两个或多个网络相互连接时,它们构成一个互联网络,或者说网际网。

正如我们之前讨论过的,一个网际网是来嗯个或多个可以互相通信的网络。最值得注意的网际网是因特网,它由成千上万个互连的网络组成。

我们已经展示了因特网通过连接设备将大大小小的网络相互交织在一起构成的基本结构。然而,如果仅仅将这些部分连接在一起,很明显什么都不会发生。为了产生沟通,即需要硬件也需要软件设备。这就像当进行一个复杂的计算时,我们同时需要计算机和程序。

当讨论因特网时,有一个词我们总是胡听见,这个词就是协议。协议定义了发送器、接收器以及所有中间设备必须遵守u以保证有效地通信的规则。简单的通信可能只需要一条简单的协议,当通信便得复杂时,可能需要将任务分配到不同的协议层中,在这种情况写,我们在每一个协议层都需要一个协议,或者协议分层。

为了更好地理解协议分层的必要,首先我们开发一个简单的场景。假定Ann和Maria是有很多共同想法的邻居,她们每次都会为了一个何时退休的计划互相见面和进行沟通。突然,Ann所在的公司为她提供一个升职的机会,但是同时要求她搬到离Maria很云的一个城市中的分部去住。由于她们想出了一个具有创新型的计划——在退休后开始做新的生意,这两个朋友仍然向继续它们的通信并且就这个新计划交换想法。她们决定通过到邮局使用普通邮件通信来继续她们的对话,但是如果邮件被拦截了,她们不想别人知晓它们的想法。...
