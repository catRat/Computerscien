\chapter{计算机网络和因特网}
个人计算机的发展带动了商业、工业、科学和教育的巨大改变。网络也发生了类似的变革。技术的进步使得通信线路能传送更多、更快的信号。而不断发展的服务使得我们能够使用这些扩展的能力。计算机网络的研究导致了新技术的产生——在全球各个地方交换文本、音频和视频等数据,在任何时候快速、准确地下载或上载信息。
\section{引言}
虽然本章的目标是讨论因特网,一个将世界上几十亿台计算机互相连接的系统,我们对互联因特网的认识不应该是一个单独的网络,而是一个网络结合体,一个互联网络。因此,我们的旅程将从定义一个网络开始。然后我们将要展示如何通过网络连接来建造小型的互联网络。最终我们会展示因特网的结构并且在本章的以下部分开启研究因特网的大门。
\subsection{网络}
网络是一系列可用于通信的设备相互连接构成的。在这个定义里面,一个设备可以是一台主机,比如一台大型计算机、台式机、便携式计算机、工作站、手机或安全系统。在这种定义中,设备也可以是一个连接设备,比如用来将一个网络与另一个网络相连接的路由器,一个将不同设备连接在一起的交换机,或者一个用于改变数据形式的调制器,等等。在一个网络中,这些设备都通过有线或无线传输媒介互相连接。当我们在家通过即插即用路由器连接两台计算机时,虽然规模很小,但已经建造了一个网络。
\begin{enumerate}
	\item 局域网

	局域网通常是与单个办公室、建筑或校园内的几个主机相连的私有网络。基于机构的需求,一个局域网既可以简单到某人家庭办公室中的两台个人计算机和一台打印机,也可以扩大至一个公司范围,并包括音频和视频设备。在一个局域网中的每一台主机都作为这台主机在局域网中唯一定义的一个标识和一个地址。一台主机向另一台主机发送的数据包中包括源主机和目标主机的地址。

	\item 广域网

	广域网也是通信设备互相连接构成的。但是广域网与局域网之间有一些差别。局域网的大小通常是受限的,跨越一个办公室、一座大楼或一个校园;而广域网的地理跨度更大,可以跨越一个城镇、一个州、一个国家,甚至横跨世界。局域网将主机互联,广域网则将交换机、路由器或调制解调器之类的连接设备互连。通常,局域网为机构私有,广域网则由通信公司创建并运营,并且租给使用它的机构。我们可以看到广域网的两种截然不同的案例:点对点广域网和交换广域网。

	点对点广域网是通过传输媒介连接两个通信设备的网络。

	交换广域网是一个有至少两个端的网络。就像我们很快就会看到的那样,交换广域网用于今天全球通信的骨干网。我们也可以这么说,交换广域网是一个点对点广域网通过开关连接产生的结合体。

	\item 互联网络

	现在很难看见独立存在的局域网或广域网,它们现在都是互相连接的。当两个或多个网络相互连接时,它们构成一个互联网络,或者说网际网。
\end{enumerate}
\subsection{因特网}
正如我们之前讨论过的,一个网际网是两个或多个可以互相通信的网络。最值得注意的网际网是因特网,它由成千上万个互连的网络组成。图展示了因特网的一个概念图像。

这幅图将因特网展示为几个骨干网、供应商网络和客户网络。在顶层,骨干网位通信公司所拥有,这些骨干网通过一些复杂的交换系统相互连接。我们把这些交换系统称为网络对等交汇点(peering point)。在第二层,有一些规模较小的网络,这些网络成为供应商网络,他妈呢付费使用骨干网上的一些服务。这些供应商网络与骨干网相连,有时也连接其他应用商网络。在因特网的边缘有一些真正使用基于因特网的服务的网络,这些网络是客户网络,它们向供应商网络付费来得到服务。

骨干网和供应商网络也被称为因特网服务供应商(ISP),骨干网通常被称为国际因特网服务提供商,供应商网络则被称为国内或地域性因特网服务提供商。
\subsection{硬件和软件}

我们已经展示了因特网通过连接设备将大大小小的网络相互交织在一起构成的基本结构。然而,如果仅仅将这些部分连接在一起,很明显什么都不会发生。为了产生沟通,即需要硬件也需要软件设备。这就像当进行一个复杂的计算时,我们同时需要计算机和程序。

\subsection{协议分层}

当讨论因特网时,有一个词我们总是会听见,这个词就是协议。协议定义了发送器、接收器以及所有中间设备必须遵守以保证有效地通信的规则。简单的通信可能只需要一条简单的协议,当通信便得复杂时,可能需要将任务分配到不同的协议层中,在这种情况下,我们在每一个协议层都需要一个协议,或者协议分层。

\begin{enumerate}
	\item 情景

	为了更好地理解协议分层的必要,首先我们开发一个简单的场景。假定Ann和Maria是有很多共同想法的邻居,她们每次都会为了一个何时退休的计划互相见面和进行沟通。突然,Ann所在的公司为她提供一个升职的机会,但是同时要求她搬到离Maria很远的一个城市中的分部去住。由于她们想出了一个具有创新型的计划——在退休后开始做新的生意,这两个朋友仍然向继续它们的通信并且就这个新计划交换想法。她们决定通过到邮局使用普通邮件通信来继续她们的对话,但是如果邮件被拦截了,她们不想别人知晓它们的想法。她们在邮件的加密/解密方式上达成了一致,寄信人对邮件进行加密,这样对于入侵者而言邮件是无法阅读的;同时收件人对邮件进行解密以得到原始邮件。第16章将讨论加密/解密方法,但是现在我们可以推测Maria和Ann使用了其中一种使没有密钥的人很难那对邮件进行解密的技术。这样,可以把Maria和Ann之间的通信分为三个协议层,如图所示。假设Ann和Maria都各自拥有三台机器(或机器人)来完成每一个协议层的任务。

	让我们假设Maria将第一封邮件寄给Ann。Maria假设第三协议层机器是正在听她说话的Ann并对其说话,第三层协议层的机器听她说话并且创作出明文,并被传到第二层的机器,第二层的机器对文本进行加密,将它创作成密文,并传送到第一协议层的机器。这个第一协议层的机器,也许是一个机器人,把密文装进信封,加上发信人地址,然后将信寄出。

	在Ann这边,第一协议层的机器从Ann的邮件中取出邮件并通过发信人地址找出来自Maria的这一封。它从信封中取出密文并传递给第二协议层的机器,第二协议层的机器对密文进行解密,创作出明文并传递给第三协议层的机器,第三协议层的机器接受明文并且将它读出来,就像Maria在说话一样。

	协议分层使我们将大任务化简成几个更小、更简单的任务。例如,在上述中,我们可以只用一台机器来完成三台机器全部的工作,但是,如果Maria和Ann觉得机器完成的现有加密或解密无法保证她们保密,他们呢需要替换整个机器。她现在的情况下,她们只需要替换第二协议层的机器就足够了,另外两层可以保持不变;这称为模块化。在这里模块化指的是独立的协议层。一个协议层(模块)可以定义位一个具有输入和输出而不需要考虑输入时如何变成输出的黑匣子。当乡两台机器提供相同输入得到相同输出时,它们就可以相互替换。

	协议分层的一个优势就是可以将服务和其实施分开来。每层使用更底层的服务,并向较高一层提供服务;并且我们不需要考虑该层时如何实施的。例如,Maria也许可以自己完成第一协议层的工作而决定不买第一协议层的计算机(机器人)。只要Maria可以完成第一协议层提供的工作,这个通信系统就可以正常的双向运行。

	协议层的另一个优势,虽然可在简单的例子中无法看出,但是在我们讨论因特网中的协议分层时会表现出来,因为通信系统往往不仅仅具有两个端系统,还有一些只需要几个协议层而不是所有协议层的中间系统。如果我们不使用协议分层,整个系统会变得更复杂,因为那样我们得把每一个中间系统都变得和端系统一样复杂。

	协议分层有什么劣势吗?也许有人会认为单一协议层可以是整个工作变简单,而且没有每个协议层都使用低一级协议层的服务并向高一级协议层提供服务的必要。但是,就像之前提到的那样,一旦密码被破解,她们每一人得将整个机器换成新的而不是仅仅更换第二协议层。
	\item 协议分层的原则

	我们讨论一下分层的原则。第一原则规定,如果我们想达到双向通信,我们需要保证每一个协议层都可以进行两个对立且方向相反的工作。

	\item 逻辑连接

	在遵循以上两条原则之后,我们可以如图所示理解每个协议层之间的逻辑连接。这说明了层与层之间通信的存在。Maria和Ann可以认为她们能够发送该协议层创作的对象是基于每层的逻辑(假想的)连接。逻辑连接的概念会帮助我们更好地理解在数据通信和建立数据关系的网络中遇到的分层工作。

\end{enumerate}
\subsection{TCP/IP协议族}
通过第二个场景我们了解了协议分层和协议层之间的逻辑通信,这样就可以介绍传输控制协议/网际协议(TCP/IP)。如今因特网中使用的协议集(一组通过不同分层进行组织的协议)被称为TCP/IP协议族。TCP/IP协议族是一个分层协议,它由提供特定功能的交互式模块组成。分层这个术语说明每一个高层协议都基于一个或多个低层协议提供的服务。TCP/IP协议族被定义成图所示布局的软件层。
\begin{enumerate}
	\item 分层架构

	为了展示TCP/IP协议族中的分层是如何在两台主机通信中作用的,我们假设要使用的一套痛惜是在一个由3个LAN(连接)构成的小网络,且链路层开关与每个LAN相连。同时我们假定这些链接都与同一个路由器相连。

	路由器只涉及三层,由于路由器仅用来路由,所以在路由器中没有传输层或应用层。

	\item 地址和数据包名称

	在因特网中,另外两个和协议分层有关的概念很值得一提,它们是地址和数据包名称。就像我们在之前讨论过的那样,在这个模型中,我们由层组之间的逻辑通信。任何涉及两步 校验的通信需要两个地址:源地址和目标地址。虽然看上去我们需要5组地址,每个协议层一组,但是正常情况下我们只有4组,因为物理层不需要地址。这是由于物理层数据交换的单位是位,这使它无法得到地址。
\end{enumerate}

	在应用层,我们通常使用名称来第一提供服务的站点,比如someorg.com,或者邮箱地址,比如somebody@coldmail.com。在传输层,地址被称为端口号,这些端口号的作用是在源和目的之间定义应用层程序。端口号的作用是通过各程序的本地地址来辨别多个同时运行的本地程序。在网络层,这些地址在整个因特网范围下是全球化的,网络层的地址独一无二得顶一个了该设备与因特网的连接。链路层地址,有时称为MAC地址,是在本地定义的地址,每一个链路层地址在计算机网络或广域网中定义一个特定的主机或者路由器。
\section{应用层}
在简单讨论网络、互联网和因特网之后,我们准备好了来进行一些关于TCP/IP协议族中每一层的讨论。我们从第5层像第1层讨论。

TCP/IP协议的第5层叫做应用层。应用层像用户提供服务。通信由逻辑连接提供,也就是说,假设两个应用层通过之间假想的直接连接发送和接受消息。

\subsection{提供服务}
应用层与其他层不同的地方在于,它是协议族中的最高层。在这层中的协议不向其他协议提供服务,它们只接收在传输层的协议提供的服务。这意味着该层的协议可以轻易去除。只要新的协议可以使用传输层中任意一个协议提供的服务,这个新的协议就可以添加到应用层上。

因为应用层时唯一向因特网用户提供服务的层,应用层的灵活性使得新的协议可以很容易地添加到网络上,如前所述,这个情况在英特网地生命周期内常常出现。当因特网最初创造出来时,用户只能使用很少地应用协议,如今由于新协议地添加已经变成常态,我们很难弄给出现有协议地具体数目。

\subsection{应用层模式}
很清楚地一点是,当使用网络时我们需要两个应用程序彼此交互:其中一个应用程序在世界上某处地一台计算机上运行,另一个世界上某处的另一台计算机上运行。这两个程序需要通过网络基础设施相互发送消息。然而,我们还没有讨论这些程序之间应该是何种关系。两个应用程序都应能够请求和提供服务,或者这些应用程序只需要其中的一个或另外一个功能?在网络的生命周期中,应用程序发展出了两种模式来解答这个问题:客户机-服务器模式和端到端模式。这里简单介绍这两种模式。

\begin{enumerate}
	\item 传统模式:客户机-服务器模式

		较为传统的模式叫做客户机-服务器模式。这直到几年以前还一直是最为流行的一种模式。在这种模式中,服务器提供者是一个应用程序,叫做服务器进程,这个进程一直持续运转,等待另一个叫做客户端进程的应用程序通过因特网连接要求服务。通常一些服务器进程可以体哦概念股某特定种类的服务,但是向这些服务器进程请求服务的用户会很多,因此很多服务器进程需要一直运行,而客户端进程只在需要时运行。

		虽然在客户机-服务器模式中的通信在这两个应用程序之进行的,但每个应用程序的角色完全不同,也就是说,我们不嫩那个把客户端程序当成服务器程序运行,反之亦然。
		这种模式的一个问题时通信负荷击中由服务器承担,这说明服务器必须时一台极为强大的计算机。即使时一台几位强大的计算机也可能因为大量客户子啊同一时间尝试连接到服务器而过载。另外一个问题则是需要服务供应商愿意接受这个成本并未某一特定服务器建造一台足够强大的服务器。也就是说,该服务必须向服务器回报相应的某种收入来鼓励这样一种安排。

		一些传统服务仍然在使用这种模式,包括万维网和它的超文本传输协议、文本传输协议、安全外壳协议、邮件服务,等待。本章后面讨论这些协议和应用。
	\item 新模式:端到端模式

		端到端模式是一个新的模式,这种模式为了响应一些新应用的需求而形成。在这种模式中,不需要一个已知运行并等待客户端进程连接服务器进程。这个责任是在端与端之间共享的。一台与网络向连接的计算机可以在一个时间段体提供服务又在另一个时间段接收服务。
\end{enumerate}
\section{标准化客户机-服务器应用}
在网络生命周期中,发展出了几种客户机-服务器应用程序。不需要重新定义它们,但是需要理解这些程序的作用。
\begin{enumerate}
	\item 万维网和超文本传输协议

		Web是具有连接分布在世界各地的文本中信息的存储库。这个存储库中叫做网页的文档分布在全世界并且相关的文档都链接在一起。web的普及和发展与以上提到的两个术语相关:分布式和链接。分布式促进了web的发展,世界上的每一个web服务器都可以往这个存储库上添加一个新网页并且告诉所有网络用户而不用担心导致个别服务器过载。链接使一个网页可以参考存储在世界上另一个敌方的服务器中的另一个网页。网页之间的链接使通过一个叫做超文本的概念达到的,这个概念的引入发生在网络到来之前。这个想法是当文档中出现到另外一个文档的链接时,用一个可以自动索引的及其检索系统中存储的另外一个文档。web将这个想法电子化了:当用户点击链接时,就会检索到被链接的文档。现在我们将为了描述链接的文本文档而创造的术语超文本改成超媒体,来说明一个网页可以是文本文档、图像、音频文件或视频文件。

		今天的WWW是一个分布式客户机-服务器服务,在这个服务中,客户可以通过浏览器来访问使用服务器的服务。但是,提供的服务分布在许多敌方,称为站点。每个站点存储的一个或多个文档称为网页。然而,每个网页都包含相同的站点或不同站点的其他网页的来链接。每个网页都是一个具有名称和地址的文件。
	\item 客户端(浏览器)

		各种各样的供应商提供了能解释和显示网页的商用浏览器。它们几乎使用了相同的体系结构。每个浏览器哦通常由三部分构成:控制器、客户端协议和解释器。

		控制器接收来自键盘或鼠标的输入,使用客户端程序存取文档。在文档被存取之和,控制器使用一个解释器在屏幕上显示温度。客户端协议可以是稍后要描述的协议中的一种,如HTTP或FTP。根据文档类型,解释器可以是HTML、Java或JavaScript。一些商业浏览器包括。
	\item 服务器

		服务器存储网页。每当请求到达时,相应的文档会发送☞客户端。
	\item 同意资源定位器(URL)

		作为文件,网页需要唯一地址标识符来将它和其他问也区分开来。定义一个网页需要3个标识符:主机、端口和路径。然而,在对网页进行定义之前,需要高数浏览器我们想要使用的客户机-服务器应用程序,这个就叫做协议。这意味着我们需要4个标识符来定义一个网页,第一个是用来得到网页的工具种类,剩下三个的组合定义目标对象(网页)。
	\item 超文本传输协议

		超文本传输协议(HTTP)是一个用来定义如何编写客户机0服务器程序以便于从网络中检索网页的协议。HTTP客户端发送请求,服务器返回响应。服务器使用的端口号为80,而客户机使用临时端口号。
\end{enumerate}
\section{文本传输协议}
文件传输协议(FTP)是TCP/IP提供的标准协议,用于从一台计算机复制文件到另一台计算机。虽然从一个胸痛到另一个胸痛文件传输看起来简单直接,但有些问题必须首先处理。例如,两个胸痛可能使用不同的文件命名约定。两个系统也可能有不同的方式表示数据。两个系统有不同的目录结构。所有这些问题都被FTP使用非常简单又没的方式解决了。
\section{电子邮件}
电子邮件(electronic mail)运行客户交换信息。

