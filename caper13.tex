
在这一章中,我们将讨论文件的结构。基于不同的应用,使用多种方法,文件被存储在辅助存储设备中。我们还将讨论当个记录是如何被检索的。本章是下一章的前序,下一章将讨论相关文件的集合是如何组织和存取的。

文件存储在辅助存储设备或二级存储设备中。两种最常见的二级存储设备是磁盘和磁带。文件在二级存储设备中是可读写的。文件也可以以计算机只能写不能读的形式存在。例如,在系统监视器上显示的信息,就是一种类似于发送到打印机的数据形式的文件。广义上,键盘也是文件,虽然它并不能存储数据。

按照我们的意图,文件是数据记录的集合,每一个记录都由一个或多个域组成。就像第11章中定义的一样。

再设计一个文件时,关键问题是如何从文件中检索信息。有时需要一个接一个地处理记录,有时又需要快速存取一个特定的记录而不用检索前面的记录。存取的方法决定了怎样检索记录:顺序或者随机的。

如果需要顺序地存取记录,则使用顺序文件结构。

如果想存取某一个特定记录而不用检索其之前的所有记录,则使用允许随机存取的文件结构。有两种文件结构都允许随机存取:索引文件和散列文件。文件结构的分类方法参见图。

顺序文件是指记录只能按照顺序从头到尾一个接一个地进行存取。记录被一个接一个地存储到辅助存储器中,并在最后的记录加上EOF的标志。操作系统没有有关记录地址的信息,它只知道记录是一个挨着一个存取的。

顺序文件用于需要从头到尾存取记录的应用。例如,如果公司里每个职员个人资料存储于一个文件中,月底就可以通过顺序存取检索每条记录来打印工资。在这里,因为必须处理每一条记录,所以顺序存取比随机存取更简便有效。

顺序文件必须定期更新,已反映信息的变化。更新过程将非常棘手,因为所有记录都要被顺序地检查和更新。

和更新程序有关的一共有4个文件:新主文件、旧主文件、实务文件和错误报告文件。所有这些文件根据关键字值被分类。虽然我们用了磁带的符号表示文件,但是我们可以更容易永磁盘的符号表示。

要使文件更新过程有效率,所有文件都必须按同一键排序。

在文件中随机存取记录,需要知道记录的地址。例如,一个客户想要查询银行账户。客户和出纳员都不知道客户记录的地址。客户只能给出纳员自己的账号。这里,索引文件可以把账号和记录地址关联起来。

目录是大多数操作系统提供的,用来组织文件。目录完成的功能就像是档案柜中的文件夹一样。但是,在大多数操作系统中,目录被表示为含有其他文件信息的一种特殊文件类型。目录的作用不仅仅像一种索引文件,该索引文件告诉操作系统文件在辅助存储设备上的位置,目录还包含了关于它所包含的文件的其他信息,如:谁有访问文件的权限,文件被创建、存取和修改的日期。

在大多操作系统中,目录被组织成像树的抽象数据类型(ADT),这种抽象数据类型在第12章中讨论过,其中除根目录外每个目录都有双亲。包含在另一个目录中的目录称为包含目录的子目录。

根目录

主目录

工作目录

父目录

文件系统的每个目录和文件都必须有一个名字。

在结束这章之前,我们讨论两个用于文件分类的术语:文本文件和二进制文件。存储在存储设备上的文件是一个位的序列,可被应用程序翻译成一个文本文件或是二进制文件。

文本文件是一个字符文件。在他们的内存储器格式中不能包含整数、浮点数或其他数据结构。要存储这些类型的数据,必须把它们转换成对应的字符格式。

一些文件只能用字符数据格式。值得注意的是用于键盘、监视器和打印机的文件流。这也是为什么需要特殊的函数来格式化上述设备的输入或输出数据。

让我们来看这样一个例子,当数据(文件流)传给打印机时,打印机获得8位数据,解释成1字节。将其译码成打印机的编码系统(ASCII或EBCDIC),如果字符处于可打印的种类,它将被打印;否则,将执行另外的操作,如打印空白。然后,打印机将读取下一个8位数据,一次重复该过程知道数据流结束。

二进制文件是计算机的内部格式存储的数据集合,在这种定义中,数据可以是整型(包括其他表示成无符号整数的数据类型,例如图像、音频或视频)、浮点型或其他数据结构(除了文件)。

不想文本文件,二进制文件中的数据只有当被程序正确地解释时才有意义。如果数据是文本格式的,就用一个字节来表示一个字符。如果数据是数字格式的,则用两个或是更多字节来表示一个数据项。
