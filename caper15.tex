
进来,技术改变了我们传输和存储数据的方式。例如,光纤电缆使我们能更加快速的传输数据,DVD技术使得在较小物理媒介上存储大量的数据成为可能。然而,如同生活的其他方面一样,人么的要求也正逐步增加。今天,人们希望在更短的事件内才在更多的数据。同样人们也希望在更小的空间存储更多的数据。 \par
压缩数据通过部分消除数据中内在的冗余来减少发送或存储的数据量。当我们产生数据的同时,冗余也就产生了。通过数据压缩,提高了数据传输和存储的效率,同时保护了数据的完整性。 \par
数据压缩意味着发送或存储更少的位数。虽然有很多放啊用于此目的,但这些方法一般可分为两类:无损压缩和有损压缩。 \par
在无损压缩中,数据完整性是受到保护的。原始数据与压缩和解压后的数据完全一样。因为在这种压缩方法中,压缩和解压算法是完全互反的两个过程,在处理过程中没有数据丢失。冗余数据在压缩时被移走,在解压时再被加回来。
