\chapter{数据存储}
正如第1章所述,计算机是一个可编程的数据处理机器。在谈论处理数据之前,我们需要理解数据的特性。在本章,我们讨论不同的数据类型以及他们呢是如何存储在计算机中的。第4张讲解计算机内部是如何控制数据的。
\section{数据类型}
如今,数据以不同的形式出现,如:数字、文本、音频、图像和视频。人们需要能够处理许多不同的数据类型:
\begin{enumerate}
	\item 工程程序使用计算机的主要目的是处理数字:进行算术运算、求解代数或三角方式、找出微分方程的根等。
	\item 与工程程序不同的是,文字处理程序使用计算机的主要目的是处理文本:调整对齐、移动、删除等。
	\item 计算机同样也处理音频数据。我们使用计算机播放音乐,并且可以把音乐作为数据输入到计算机中。
	\item 图像处理程序使用计算机的主要目的是处理图像:创建、收缩、放大、旋转等。
	\item 最后,计算机不仅能用来播放电影,还能创建我们在电影中所看到的特技效果。
\end{enumerate}
计算机行业中使用术语“多媒体”来定义包含数字、文本、图像、音频和视频的信息。

\section{计算机内部的数据}
所有计算机外部的数据类型的数据都采用同意的数据表示法转换后存入计算机中,当数据从计算机输出时再还原回来。这种通用的格式成为位模式。
\begin{enumerate}
	\item 位

	位(bit,binary digit的缩写)是存储在计算机中的最小单位;它是0或1.位代表设备的某一状态,这些设备只能处于两种状态之一。例如,开关要么合上要么断开。用1表示合上状态,0表示断开状态。电子开关就表示一个位。换句话说,一个开关能存储一个位的信息。今天,计算机使用各种各样的双态设备来存储数据。
	\item 位模式

	为了表示数据的不同类型,应该使用位模式,它是一个序列,有时也被称为位流。这就意味着,如果我们需要存储一个由16个位组成的位模式,那么需要16个电子开关。如果我们需要存储1000个位模式,每个16位,那么需要16 000个开关。通常长度位8的位模式被称为1字节。有时用字这个术语指代更长的位模式。

	正如图所示,属于不同数据类型的数据可以以同样的模式存储于内存中。

	如果使用文本编辑器,键盘上的字符A可以以8位模式01000001存储。如果使用数学程序,同样的8位模式可以表示数字65.类似地,同样的位模式可表示部分图像、部分歌曲、影片中的部分场景。计算机内存存储所有这些而无需辨别它们表示的是何种数据类型。
	\item 数据压缩

	为占用较少的内存空间,数据在存储到计算机之前通常被压缩。数据压缩是一个广阔的主题,所以我们用整个15章来讲述。
	\item 错误检测和纠正

	另一个与数据有关的话题是在传输和存储数据时的错误检测和纠正。
\end{enumerate}
\section{存储数字}
在存储到计算机内部中之前,数字被转换到二进制系统。但是,这里还有两个问题需要解决:
\begin{enumerate}
	\item 如何存储数字的符号。
	\item 如何显示十进制小数点。
\end{enumerate}
有多种方法可处理符号问题,本章后面陆续讨论。对于小数点,计算机使用两种不同的表示方法:定点和浮点。第一种用于把数字作为证书存储——没有小数部分,第二种把数字作为实数存储——带有小数部分。
\subsection{存储整数}
整数是完整的数字(即没有小数部分)。例如,134和-125是整数而134.23和-0.235则不是。整数可以被当作小数点位置固定的数字:小数点固定在最右边。因此,定点表示法用于存储整数,如图3-4所示。在这种表示法中,小数点是假定的,但并不存储。

但是,用户可能将整数作为小数部分为0的实数存储。这是可能发生的,例如,整数太大以至于无法定义位整数来存储。为了更有效地利用计算机内存,无符号和有符号的整数在计算机中存储方式是不同的。
\begin{enumerate}
	\item 无符号表示法

	无符号整数是只包括零和正数的非负整数。它的范围介于0到无穷大之间。然而,由于计算机不可能表示这个范围的所有整数,通常,计算机都顶一个了一个常量,称为最大无符号整数,它的值是(
	$ 2^n-1$
	)。这里
	$n$
	就是计算机中分配用于表示无符号整数的二进制位数。
	\begin{itemize}
		\item 存储无符号整数

		输入设备存储无符号整数使用以下步骤:
		\begin{itemize}
			\item 首先将整数变成二进制数。
			\item 如果二进制位数不足n位,则在二进制数的左边补0,使它的总位数为
			$n$
			位。如果位数大于
			$n$
			,该整数无法存储。导致溢出的情况发生,我们后面要讨论。
		\end{itemize}
		\item 翻译解释无符号整数

		输出设备翻译解释内存中位模式的位串并转换位一个十进制的无符号整数。
		\item 溢出

		因为大小的限制,可以表达的整数范围是有限的。在
		$n$
		位存储单元中,我们可以存储的无符号整数仅为0到
		$2^n-1$
		之间。

		\item 无符号整数的应用

		无符号整数表示法可以提高存储的效率,因为不必存储整数的符号。这就意味着所有分配的位单元都可以用来存储数字。只要用不到负整数,都可以用无符号整	数表示法。具体情况如下:
		\begin{itemize}
			\item 计数:当我们计数时,不需要负数。可以从1开始增长。
			\item 寻址:有些计算机语言,在一个存储单元中存储了另一个存储单元的地址。地址都是从0开始到整个存储器的总字节数的正数,在这里同样也不需要用到负数。因此无符号整数可以轻松地完成这个工作。
			\item 存储其他数据类型:我们后面讲谈到的其他数据类型是以位模式存储的,可以翻译位无符号整数。
		\end{itemize}

	\end{itemize}
	\item 符号加绝对值表示法

	尽管符号叫绝对值格式在存储整数中并不常用,但该格式用于在计算机中存储部分实数,正如下一节所述。因此,我们在这里简要讨论该格式。在这种方法中,用于无符号整数的有效范围(0到
	$2^n-1$
	)被分成两个相等的自范围。前半个表示正整数,后半个表示负整数。

	用符号加绝对值格式存储一个整数,需要用1个二进制表示符号。这就意味着在一个8位存储单元中,可以仅用7位表示数字的绝对值。因此,最大的正数值仅是无符号最大数的一半。在
	$n$
	位单元可存储的数字范围是
	$-(2^(n-1)-1)$
	至
	$+(2^(n-1) - 1)$
	,
	$n$
	位单元最左位分配用于存储符号。
	\item 二进制补码表示法

	几乎所有的计算机都使用二进制补码表示法来存储位于
	$n$
	位存储单元中的有符号整数。这一方法中,无符号整数的有效范围(0到
	$2^n-1$
	)被分为两个相等的子范围。
\end{enumerate}
\section{3种系统的比较}
\section{实数}
实数是带有整数部分和小数部分的数字。例如,23.7是一个实数——整数部分是23而小数部分是7/10.尽管固定小数点的表示法可用于表示实数,但结果不一定精确或达不到需要的精度。

\begin{enumerate}
	\item 浮点表示法

	用于维持正确度和精度的解决办法是使用浮点表示法。
\end{enumerate}
\section{存储文本}
在任何语言中,文本的片段是用来表示该语言中某个意思的一系列符号。例如,在英语中使用26个符号(A,B,C,\cdots,Z)来表示大写字母,26个符号表示小写字母,10个符号(0, 1,2,,9)来表示数字符号,以及符号(.,?,:,...,!)表示标点。另外一些符号(如空格、换行和制表符)被用于文本的对齐和可读性。

我们可用位模式来表示任何一个符号。换句话说,如4个符号组成的文本“CATS”能用4个
$n$
位模式表示,任何一个模式定义一个单独的符号。

现在的问题是:在一种语言中,位模式到底需要多少位表示一个符号?这主要取决于该语言集中到底有多少不同的符号。例如,如果要创建的某个虚拟语言仅仅使用大写英文字母,则只需要26个符号。相应地,这种语言的位模式至少需要表示26个符号。

对另一种语言,如中文,科恩那个需要更多的符号。在一种语言中,表示某一符号的问模式的长度取决于该语言中所使用的符号的数量。更多的符号意味着更长的位模式。

尽管位模式的长度取决于符号的数量,但是他们的关系并不是线性的,而是对数的。如果需要2个符号,位模式长度将是1位,如果需要4个符号,长度将是2位。同样,3位的位模式有8中不同的形式。

不同的位模式集合被设计用来表示文本符号。其中每一个集合我们称之为代码。表示符号的过程被称为编码。在这个部分将介绍常用代码。

\begin{enumerate}
	\item ASCII

	美国国家标准协会(ANSI)开发了一个被称为美国信息交换标准(ASCII)的代码。该代码使用7位表示每个符号。即改代码可以定义
	$2^7=128$
	种不同的符号。如今ASCII是Unicode的一部分,下面将要讨论。
	\item Unicode

	硬件和软件制造商联合起来共同设计了一种名为Unicode的代码,这种代码使用32位并能表示最大达
	$2^32=4 294 967 296$
	个符号。代码的不同部分被分配用于表示来自世界上不同语言的符号。其中还有些部分被用来表示图形和特殊符号。
	\item 其他编码

	最近几十年来,其他编码不断发展。鉴于Unicode的优势,这些编码变得不太流行。我们把了解这些编码作为练习。
\end{enumerate}
\section{存储音频}
音频表示声音或音乐。音频本质上与我们讨论到现在的数字和文本是不同的。文本由可数的实体(文字)组成:我们可以数出文本中文字的数量。文本是数字数据的一个例子。相反,音频是不可数的。音频是随时间变化的实体,我们只能在每一时刻度量声音的密度。当我们讨论用计算机内存存储声音时,我们的意思是存储一个音频信号的密度,例如,每隔一段时间来自麦克风的信号。

音频是模拟数据的例子。即使我们能够在一段时间度量所有的值,也不能把它全部存在计算机内存中,因为可能需要无限数量的存储单元。

\subsection{采样}
如果我们不能记录一段间断的音频信号的所有值,至少我们可以记录其中的一些。采样意味着我们在模拟信号上选择数量有限的点来度量它们的值并记录下来。图显示了从这些信号上选择10个样本:我们可以记录这些值来表现模拟信号。
\subsection{采样率}
下一个逻辑问题是,我们每秒需要多少样本才能还原出原始信号的副本?样本数量依赖于模拟信号中变化的最大数量。如果信号是平坦的,则需要很少的样本;如果信号变化 剧烈,则需要更多的样本。每秒40 000个样本的采样率对音频信号来说是足够好的。
\subsection{量化}
从每个样本测量来的值是真实的数字。这意味着我们可能要位每一秒的样本存储40 000个真实的值。但是,为了每个样本使用一个无符号的数会更简便。量化指的是将样本的值截取为最接近的整数值的一种过程。例如,如果实际的值为17.2,就可截取为17;如果值为17.7,就可截取为18.
\subsection{编码}
下面的任务是编码。量化的样本值需要被编码成为位模式。一些系统为样本赋正值或负值,另一些仅仅移动曲线得到正的区间从而只赋正值。换言之,一些系统使用无符号整数来表示样本,而另一些使用有符号的整数来做。但是,有符号的整数不必用在二进制补码中,它们可以用符号加绝对值的值。最左边的位用于表示符号,其余的位用于表示绝对值。
\begin{enumerate}
	\item 每样本位

	对于每个样本系统需要决定分配多少位。尽管在过去仅有8位分配给声音样本,现在每个样本16、24甚至32位都是正常的。没样本位的数量有时称为位深度。
	\item 位率

	如果我们称位深度或每样本位的数量位B,每秒样本数位S,我们需要每秒的音频存储SXB位。该乘积有时称为位率R。例如,如果我们使用每秒40 000个样本以及没样本16位,位率是R=40 000X16=640 000b/s=640KB/s。
\end{enumerate}
\subsection{声音编码标准}
当今音频的主流标准是MP3(MPEG Layer3的缩写)。该标准是用于视频压缩方法的MPEG(动态图像专家组)标准的一个修改版。它采用每秒44 100个样本以及没样本16位。结果信号达到705 600b/s的位率,再用去掉那些人耳无法识别的信息的压缩方法进行压缩。这是一种有损压缩法,与无损压缩法相反。
\section{存储图像}
存储在计算机中的推向使用两种不同的计数:光栅图或矢量图。
\subsection{光栅图}
当我么需要存储模拟图像时,就用到了光栅图(或位图)。一张照片由模拟数据组成,类似于音频信息。不同的时数据密度(色彩)因空间变化,而不是因时间变化。这意味着数据需要采样。然而,这种情况下采样通常被称作扫描。样本称为像数(代表图像元素)。换言之,真个图像被分成效地像数,每个像素假定有单独的密度值。
\begin{enumerate}
	\item 解析度

	就像音频采样那样,在图像扫描中,我们要决定对于每英寸的方块或线条需要记录多少像数。在图像处理中的扫描率称为解析度。如果解析率足够高,人眼不会看出在从先图像中的不连续。
	\item 色彩深度

	用于表现像素的位的数量,即色彩深度,依赖于像素的颜色时如何由不同的编码计数来处理的。对颜色的感觉是我们的眼睛如何对光线的响应。我们的眼睛有不同类型的感光细胞:一些响应红、黄、蓝三原色,而另一些仅仅响应光的密度。
	\begin{enumerate}
		\item 真色彩

		用于像数编码的技术之一称为真彩色,它使用24位来编码一个像素。在该计数中,每个三原色都表示为8位。因为该技术中8位模式可以表示0~255之间的一个数,所以每种色彩都由0~255之间的三维数字表示。
		\item 索引色

		真彩色模式使用了超过1600万种的颜色。许多应用程序不需要如此大的颜色范围。索引色(或调色板色)模式仅使用其中的一部分。在该模式中,每个应用程序从大的色彩集中选择一些颜色并建立索引。对选中的颜色赋一个0~255之间的值。这就类似艺术家可能在它们的画室用到很多中颜色,但一次仅用到他们调色板中的一些。
	\end{enumerate}
	\item 图像编码标准

	几种用于图像编码的实际标准在使用中。JPEG使用真彩色模式,但压缩图像来减少位的数量。另一方面,GIF使用索引色模式。
\end{enumerate}
\subsetion{矢量图}
光栅图有两个缺点,即文件体积太大和重新调整图像大小有麻烦。放大光栅图像意味着扩大像素,所以方法后的图像看上去很粗糙。但是,矢量图图像编码方法并不是存储每个像数的位模式。一个图像被分解成几个几何图形的组合,例如,线段、矩形或圆形。每个几何形状由数学公式表达。例如,线段可以由它的断电的坐标描述,圆可以由它的圆心坐标和半径长度来描述。矢量图是由定义如何绘制这些形状的一系列命令构成的。

当要显示或打印图像时,将图像的尺寸作为输入传给系统。系统重新设计图像的大小并用相同的公式画出图像。在这种情况下,每一次绘制图像,公式也将重新估算一次。因此矢量图也称为几何模型或面向对象图形。

矢量图不适合存储照片图像的细微精妙。JPEG或GIF光栅图提供了更好更生动的图片。矢量图适合应用程序采用主要的集合元素来创建图像。它用于诸如Flash这样的应用程序,以及创建TrueType(微软、苹果公司)和PostScript(Adobe公司)字体。计算机辅助设计(CAD)也使用矢量图进行工程绘图。
\section{存储视频}
视频时图像(称为帧)在时间上的表示。一部电影就是一系列的帧一张接一张地播放而形成运动图像。换言之,视频时随空间和时间变化的时间表现。所以,如果知道如何将一副图像存储在计算机中,我们也就知道如何存储视频;没衣服图像或帧转化成一系列位模式并存储。这些图像组合起来就可以表示视频。需要注意心啊在视频通常是被压缩存储的。在第15章,我们将讨论MPEG,这是一种常见的视频压缩技术。
