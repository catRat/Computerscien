在前面章节中,我们使用变量来存储单个实体,尽管单变量在程序设计语言中被大量使用,但它们不能有效地解决复杂问题。本章将介绍数据结构,它是第12章首相数据结构的前奏。

数据结构利用了有关的变量的集合,而这些集合能够单独或作为一个整体被访问。换句话说,一个数据结构代表了有特殊关系的数据的集合。本章将讨论三种数据结构:数组、记录和链表,大多的编程语言都隐式实现了前两种而第三种则通过指针和记录来模拟。

假设100个分数,我们需要读入这些书,处理他们并打印。同时还要求将这100个分数在处理过程中保留在内存中。可以定义100个变量,每个都有不同的名字。

但是定义100个不同的变量名带来了其他问题,我们需要100个引用来读取它们;需要100个引用来处理它们;需要100个引用来写它们。

及时是处理这些数目相对小的分数,我们需要的指令数目也是无法接受的。为了处理大量的数据,需要一个数据结构。

数组是元素的顺序集合,通常这些元素具有相同的数据类型。虽然有些编程语言允许在一个数据中可以有不同类型的元素。我们可以成数组中的元素为第一个元素、第二个元素等,知道最后一个元素。如果将100个分数放进数组中,可以指定元素为csores[1]、scores[2],等等。索引表示元素在数组中的顺序号,顺序号从数组开始处计数。数组元素通过索引被独立给出了地址,这个数组整体上有个名称:scores,但每个数可以利用他的索引来单独访问。

我们可以使用村换来读写数组中的元素,也可以使用村换来处理元素。现在我们可以不管要处理的元素是100个、100个或10 000个,循环使得处理他们变得容易。我们可以使用一个整数变量来控制循环,只要变量的值小于数组找那个元素的个数,那还在循环体里面。

在一个数组中,有两种标识符:数组的名字和各个元素的名字。数组名是整个结构的名称,而元素的名字允许我们查阅这个元素。在上个例子中,数组的名字是scores,而每个元素的名字是这个名字后面跟索引,如scores[1]、scores[2]等。在本章中,我们大部分需要的是元素的名字,但是,有些语言中,也需要使用数组的名字。

到目前为止,我们所讨论的都是一维数组,以内数据仅是在一个方向上线性组成。许多的应用 要求数据存储在多维中。常见的例子如表格,就是包括行和列的数组。

虽然我们能应用一些为数组中每个元素定义的通常操作,但我们还能定义一些把数组作为数据结构的操作,数组作为结构的常用操作有:查找、插入、删除、检索和遍历。

当我们知道元素的值时,经常需要找到元素的序号,这种操作在第8章讨论过。我们可以对未排序的数组使用顺序查找,对排序的数组使用折半查找。下面三种操作都要使用查找操作。

通常,计算机语言要求数组的大小在程序被写的时候就被定义,防止在程序的执行过程中被修改。最近,有些语言允许可变长数组。及时语言允许可变长数组,在数组找那个插入一个元素仍需要十分小心。

如果插入操作在数组尾部进行,而且语言允许增加数组的大小,那么可以很容易完成这个操作。

如果插入是在数组的开始或中间,过程就是冗长的和花费时间的。当我们需要在一个有序的数组中插入一个元素时,这就发生了。就像前面描述的,首先查找数组。找到插入的位置后,插入新的元素。例如...

在数组中删除一个元素就像插入操作一样冗长和棘手。

检索操作就是随便地存取一个元素,达到检查或复制元素中的数据的目的。与插入和删除操作不同,当数据结构是数组时,检索是一个容易地操作。实际上,数据是随机存取结构,这意味着数组的每个元素可以随机地被存取,而不需要存取该元素前面的元素或后面的元素。

数组的遍历是指被应用于数组中每个元素上的操作,如读、写、应用数学的运算等。

考虑一下前一节所讨论的操作,就给出了数组应用的提示。如果有一个表,在表创建后有大量的插入和删除操作要进行,这是就不应该使用数组。当删除和插入操作较少,而有大量查找和检索操作时,这时比较适合使用数组。

记录是一组相关元素的集合,它们可能是不同的类型,但整个记录有一个名称。记录中的每个元素称为域。域是具有含义的最小命名数据。它有类型且存在于内存中。它能你被赋值,繁殖也能够被选择和操纵。域不同于变量主要在于它是记录的一部分。

在记录中的元素可以是相同类型或不同类型,但记录中的说有元素必须是关联的。

就像数组一样,在记录中也有两种标识符:记录的名字和记录总各个域的名字。记录的名字是整个结构的名字,而每个域的名字允许我们存取这些域。

我们可以从概念上对数组和记录进行比较。这有助于我们理解什么时候应该使用数组,什么时候应该使用记录。数组定义了元素的集合,而记录定义了元素可以确认的部分。

如果我们需要定义元素的耳机和,同时需要定义元素的属性,那么可以使用记录数组。

数组和记录数组都表示数据项的列表。数组可以被看成是记录数组的一种特例,其中每个元素是自带一个域的记录。

链表是一个数据的集合,其中每个元素包含下一个元素的地址;即每个元素包含两部分:数据和链。数据部分包含可用的信息,并被处理。链则将数据连在一起,它包含一个指明列表中下一个元素的指针。另外,一个指针变量标识该别表中的第一个元素。列表的名字就是该指针变量的名字。

链表中的元素习惯上称为节点,链表中的节点是至少包括两个域的记录:一个包含数据,另一个包含链表中下一个节点的地址。

在对链表进行进一步讨论前,我们需要解释一个在图中的几号。使用连线显示两个节点间的连接。显得一段有一箭头,线的另一端是实心圆。箭头表示箭头所指节点的地址副本。实心圆显示了地址的副本存储的地方。

数组和链表都能表示内存中的数据项列表。唯一的区别在于数据项连接在一起的方式。在记录数组中,连接工具是索引。元素scores[3]与元素scores[4]相连,因为整数4是紧跟着整数3后面的。在链表中,连接工具是指向下一元素的链。

数组的元素在内存中是一个接一个中间无间隔存储的,即列表是连续的。而链表中的节点在存储中间是有间隔的:节点的链部分包数据项“胶”在一起。换言之,计算机可以选着连续存储它们或把节点分布在整个内存中。这样有一个优点:在立案表彰进行的插入和删除操作更容易些,只需改变指向下一元素地址的指针。但是,这带来了二外的开销:链表的每个节点有一个额外的域,存放内存中下一节点的地址。

就像数组和记录一样,我们需要区分链表名和节点名。一个链表必须要有一个名字。

链表名是头指针名字,该头指针指向表中第一个节点。另一方面,节点在链表中并没有明显的名字,有的只是隐含的名字。节点的名字与指向节点的指针有关。不同的语言在处理节点与指针节点间的关系时是不同的。我们使用C语言中使用的约定。
