软件生命周期是软件工程中的一个基础概念。软件和其他的产品一样,周期性地重复着一些阶段。

软件最初由开发小组开发。通常,在它需要修改之前会使用一段时间。由于软件会发现错误、设计改变规则或公司本身发生变化,这些都导致需要经常修改软件。为长久使用考虑软件应该被修改。使用和修改,这两个步骤移植进行下去直到软件过时。“过时”意味着因效率低下、语言过时、用户需求的重大变化或其他因素而导致软件失去它的有效性。

虽然软件工程及图中的所有三个过程,但笨重我们只讨论开发过程,它在图中处于循环流程之外。在软件生命周期中,开发过程包括4个阶段:分析、设计、实现和测试。开发过程有多种模型,这里我们讨论常见的两种:瀑布模型和增量模型。

软件开发过程中一种非常流行的模型就是总所周知的瀑布模型。在这种模型中,开发过程只有一个方向的流动。这意味着前一个阶段不结束,后一个阶段不能开始。

例如,整个工程的分析阶段应该在设计阶段开始前完成。整个设计阶段应该在是现阶段前完成。

瀑布迷行既有优点,也有缺点。优点之一就是在下一个阶段开始前每个阶段已经完成。但是,普遍模型的缺点是难以定位问题:如果过程的一部分有问题,必需检验整个过程。

在增量模型中,软件的开发要历经一些列步骤。开发者首先完成整个系统的一个简化版本,这个版本表示整个系统,但不包括具体的细节。

在第二个阶段中,更多的细节被加入,而有些还没完成,然后再次测试系统。如果这时有问题,开发者知道问题出于新功能。知道现有系统工作正确后,它们才增加新的功能。这样过程一直继续下去,直到要求的功能全部加入。

整个开发过程始于分析阶段,这个阶段生成规范说明文档,这个文档说明明了软件要做什么,而没有说明如何去做。分析阶段可以使用两种独立的方法,它们依赖于是现阶段使用的是过程编程语言,还是面向对象语言。

如果是现阶段使用过程语言,那么面向过程分析就是分析阶段使用的方法。这种情况下的规格说明可以使用多种建模工具,但是我们只讨论其中少数几个。

数据流图显示了系统中数据的流动。它们使用4中符号:方形盒表示数据源或数据目的,带圆角的矩形表示过程。末端开口的矩形标识数据存储的地方,箭头表示数据流。

分析阶段使用的另一个建模工具是实体关系图。因为这个图也用于数据库的设计,所以我们将在12章中讨论。

状态图提供了另外一种有用的工具,它通常用于当系统中的实体状态在响应事件时将会改变的情况下。作为状态图的一个例子,我们显示了担任电梯的操作。当楼层按钮被按,电梯将按要求的方向移动,在到达目的地之前,它不会响应其他任何请求。

如果实现使用面向对象语言,那么面向对象分析就是分析过程使用的。这种情况下的规格说明文档可以使用多种工具,但我们只讨论其中少数几种。
