软件生命周期是软件工程中的一个基础概念。软件和其他的产品一样,周期性地重复着一些阶段。

软件最初由开发小组开发。通常,在它需要修改之前会使用一段时间。由于软件会发现错误、设计改变规则或公司本身发生变化,这些都导致需要经常修改软件。为长久使用考虑软件应该被修改。使用和修改,这两个步骤移植进行下去直到软件过时。“过时”意味着因效率低下、语言过时、用户需求的重大变化或其他因素而导致软件失去它的有效性。

虽然软件工程及图中的所有三个过程,但本章我们只讨论开发过程,它在图中处于循环流程之外。在软件生命周期中,开发过程包括4个阶段:分析、设计、实现和测试。开发过程有多种模型,这里我们讨论常见的两种:瀑布模型和增量模型。

软件开发过程中一种非常流行的模型就是众所周知的瀑布模型。在这种模型中,开发过程只有一个方向的流动。这意味着前一个阶段不结束,后一个阶段不能开始。

例如,整个工程的分析阶段应该在设计阶段开始前完成。整个设计阶段应该在是现阶段前完成。

瀑布模型既有优点,也有缺点。优点之一就是在下一个阶段开始前每个阶段已经完成。但是,瀑布模型的缺点是难以定位问题:如果过程的一部分有问题,必需检验整个过程。

在增量模型中,软件的开发要历经一些列步骤。开发者首先完成整个系统的一个简化版本,这个版本表示整个系统,但不包括具体的细节。

在第二个阶段中,更多的细节被加入,而有些还没完成,然后再次测试系统。如果这时有问题,开发者知道问题出于新功能。知道现有系统工作正确后,它们才增加新的功能。这样过程一直继续下去,直到要求的功能全部加入。

整个开发过程始于分析阶段,这个阶段生成规范说明文档,这个文档说明明了软件要做什么,而没有说明如何去做。分析阶段可以使用两种独立的方法,它们依赖于是现阶段使用的是过程编程语言,还是面向对象语言。

如果是现阶段使用过程语言,那么面向过程分析就是分析阶段使用的方法。这种情况下的规格说明可以使用多种建模工具,但是我们只讨论其中少数几个。

数据流图显示了系统中数据的流动。它们使用4中符号:方形盒表示数据源或数据目的,带圆角的矩形表示过程。末端开口的矩形标识数据存储的地方,箭头表示数据流。

分析阶段使用的另一个建模工具是实体关系图。因为这个图也用于数据库的设计,所以我们将在12章中讨论。

状态图提供了另外一种有用的工具,它通常用于当系统中的实体状态在响应事件时将会改变的情况下。作为状态图的一个例子,我们显示了单人电梯的操作。当楼层按钮被按,电梯将按要求的方向移动,在到达目的地之前,它不会响应其他任何请求。

如果实现使用面向对象语言,那么面向对象分析就是分析过程使用的。这种情况下的规格说明文档可以使用多种工具,但我们只讨论其中少数几种。

在瀑布模型中,设计阶段完成后,实现阶段就可以开始了。在这个阶段,程序员为面向过程设计中的模块编写程序或编写程序单元,实现面向对象设计中的类。在每种情况中,都有一些我们需要提及的问题。

在面向过程开发中,工程团队需要从第9章所讨论的面向过程语言中选择一个或一组语言。虽然有些语言被看成既是面向过程的,又是面向对象的语言,但通常实现只用纯过程语言,如C语言。在面向对象的情况下,C++和java的使用都很普遍。

在实现阶段创建的软件质量是一个非常重要的问题。具有高质量的软件系统是一个能满足用户需求、符合组织操作标准和能高效运行在为其开发的硬件上的一个软件。但是,如果我们需要取得高质量的软件系统,那么必须能定义质量的一些属性。

软件质量能够划分成三个广义的度量:可操作性、可维护性和可迁移性。每个度量如图所示还可以扩展。

可操作性涉及系统的基本操作。就像图中的显示一样,可操作性有多种度量方法:准确性、高效性、可靠性、安全、及时性和适用性。

不准确的系统比没有系统更糟糕。任何被开发的系统都必须经过系统测试工程师和用户检测。准确性能够通过诸如故障平均时间、每千行代码错误数以及用户请求变更数这样的测量指标来度量。

高效性大体上是个主管的术语。在一些实例中,用具将指定性指标,例如实时响应必须在1秒之内接收到,成功率在95%。它显然是可测量的。

可靠性实际上综合了其他各种因素。如果用户指望系统完成工作并对其有信心,这时它就是可靠的。另外,一些度量直接说明了系统的可靠性,最显著的故障平均时间。

一个系统的安全性是以未经授权的人得到系统数据的难易程度为参照。尽管这时一个主观的领域,但仍然有可核查的清单帮助评估系统的安全性。例如,系统有并且需要密码来验证用户吗?

在软件工程中,及时性意味着几件不同的事情。系统及时传递它的输出了吗?对于在线系统,响应时间满足用户的需求了吗?

适用性是另一个很主观的领域。对适用性的最好度量方法是观察用户,看他们是如何使用这个系统的。用户访谈能够常常发现系统适用性上的问题。

可维护性以保湿系统正常运行并及时更新为参照。很多系统需要经常修改,这不是因为它们不能拿很好地运行而是因为外部因素的改变。

可变性是个主观因素。

可修正性的只用度量是回复正常的平均时间。

可迁移性是指把数据和系统从一个平台移动到另一个平台并重用代码的能力。在很多情况下,这不是一个很重要的因素。另一方面,如果你编写具有通用性的软件,那么可迁移性就很关键了。

测试阶段的目的就是发现错误,这就意味着良好的测试策略能发现最多的错误。有两种测试:白盒测试和黑盒测试。

白盒测试是基于知道软件内部结构的。测试的目的是检验软件所有的部分是否全部设计出来。白盒测试假定测试者知道有关软件的一切。在这种情况下,程序就像玻璃盒子,其中的每件事都是可见的。白盒测试由软件工程师或一个专门的团队来完成。使用软件的白盒测试需要保证至少满足下面4条标准:

每个模块中的所有独立的路径至少被测试过一次。

所有的判断结构每个分支都被测试。

每个循环被测试。

所有数据结构都被测试。

在过去,已经有多正测试方法,我们只讨论其中的两种:基本路径测试和控制结构测试。

基本路径测试是由Tom McCabe提出。这种方法创建一组测试用例,这些用例执行软件中的每条语句至少一次。

基本路径测试使用图论和圈复杂性找到必须被走过的独立路径,从而保证每条语句至少被执行一次。

控制结构测试比基本路径测试更容易理解并且包含基本路径测试,这种方法使用下面将要简要讨论的不同类的测试。

(1)条件测试

条件测试应用于模块中的条件表达式,简单条件是关系表达式,而复合条件是简单条件和逻辑运算符的组合。条件测试用来检查是否所有的条件都被正确设置。

(2)数据流测试

数据流测试是基于通过模块的数据流。这种测试选择测试用例,这些用例涉及检查被用在赋值语句左边的变量的值。

(3)循环测试

循环测试使用测试用例检查循环的正确性。所有类型的循环都仔细测试。

黑盒测试在不知道程序的内部也不知道程序是怎样工作的情况下测试程序。换言之,程序就像看不见内部的黑盒。黑盒测试按照软件应该完成的功能来测试软件,如它的输入和输出。下面介绍几种黑盒测试方法。

最好的黑盒测试方法就是用输入域中的所有可能的值去测试软件。但是,在复杂的软件中,输入域是如此巨大,这样做常常不现实。

在随机测试中,选择输入域的值的子集来测试,子集选择的方式是非常重要的。在这种情况下,随机数生成器是非常有用的。

当遇到边界值时,错误经常发生。例如,一个模块定义它的输入必须大于或等于100,那这个模块用边界值100来写实就非常重要。如果模块在边界值出错,那有可能就是模块代码中的有些条件,例如,$x>100$被写成$x>100$。

软件的正确使用和有效维护离不开文档。通常软件有三个独立的文档:用户文档、系统文档和技术文档。注意文档是一个持续的过程。如果软件在发布之后有问题,也必须写文档。如果软件被修改,那么所有的修改和与原软件包间的关系都要被写进文档。只有当软件包过时后,编写文档才停止。

为了软件包正常运行,传统上称为用户手册的文档对用户是必不可少的。它告诉用户如何一步步地使用软件包。它通常包含一个教程知道用户熟悉软件包的各项特征。

一个好的用户手册能够称为一个功能强大的营销工具。用户文档在营销中的重要性再强调也不过分。手册应该面向新手和专业用户。配有好的用户文档必定有利于软件的销售。

系统文档定义软件本身。转系系统文档的目的是为了让原始开发人员之外的人能够维护和修改软件包。系统文档在系统开发的所有4个阶段都应该存在。

在分析阶段,收集的信息应该仔细地用文档记录。另外,系统分析员应该定义信息的来源。需求和选用的方法必须用基于它们的推论来清楚表述。

在设计阶段,最终版本中用到的工具必须记录在文档中。例如,如果结构图修改了多次,那么最终的版本要用完整的注释记录在案。

在实现阶段,代码的每个模块都应记录在文档中。另外,代码应该使用注释和描述头尽可能详细地形成自文档。

最后,开发人员必须仔细地形成测试阶段的文档。对最终产品使用的每种测试,连同它的结果都要记录在文档中。甚至令人不快的结果和产生它们的数据也要记录在案。

技术文档描述了软件系统的安装和服务。安装文档描述了软件如何按章在每个计算机上,如服务器和客户端。服务文档描述了如果需要,系统应该如何维护和跟新。
